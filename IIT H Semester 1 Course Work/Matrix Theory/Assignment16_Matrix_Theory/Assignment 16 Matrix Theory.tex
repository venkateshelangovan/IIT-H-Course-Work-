\documentclass[journal,12pt]{IEEEtran}
\usepackage{longtable}
\usepackage{setspace}
\usepackage{gensymb}
\singlespacing
\usepackage[cmex10]{amsmath}
\newcommand\myemptypage{
	\null
	\thispagestyle{empty}
	\addtocounter{page}{-1}
	\newpage
}
\usepackage{amsthm}
\usepackage{mdframed}
\usepackage{mathrsfs}
\usepackage{txfonts}
\usepackage{stfloats}
\usepackage{bm}
\usepackage{cite}
\usepackage{cases}
\usepackage{subfig}

\usepackage{longtable}
\usepackage{multirow}

\usepackage{enumitem}
\usepackage{mathtools}
\usepackage{steinmetz}
\usepackage{tikz}
\usepackage{circuitikz}
\usepackage{verbatim}
\usepackage{tfrupee}
\usepackage[breaklinks=true]{hyperref}
\usepackage{graphicx}
\usepackage{tkz-euclide}

\usetikzlibrary{calc,math}
\usepackage{listings}
    \usepackage{color}                                            %%
    \usepackage{array}                                            %%
    \usepackage{longtable}                                        %%
    \usepackage{calc}                                             %%
    \usepackage{multirow}                                         %%
    \usepackage{hhline}                                           %%
    \usepackage{ifthen}                                           %%
    \usepackage{lscape}     
\usepackage{multicol}
\usepackage{chngcntr}

\DeclareMathOperator*{\Res}{Res}

\renewcommand\thesection{\arabic{section}}
\renewcommand\thesubsection{\thesection.\arabic{subsection}}
\renewcommand\thesubsubsection{\thesubsection.\arabic{subsubsection}}

\renewcommand\thesectiondis{\arabic{section}}
\renewcommand\thesubsectiondis{\thesectiondis.\arabic{subsection}}
\renewcommand\thesubsubsectiondis{\thesubsectiondis.\arabic{subsubsection}}


\hyphenation{op-tical net-works semi-conduc-tor}
\def\inputGnumericTable{}                                 %%

\lstset{
%language=C,
frame=single, 
breaklines=true,
columns=fullflexible
}
\begin{document}
\onecolumn

\newtheorem{theorem}{Theorem}[section]
\newtheorem{problem}{Problem}
\newtheorem{proposition}{Proposition}[section]
\newtheorem{lemma}{Lemma}[section]
\newtheorem{corollary}[theorem]{Corollary}
\newtheorem{example}{Example}[section]
\newtheorem{definition}[problem]{Definition}

\newcommand{\BEQA}{\begin{eqnarray}}
\newcommand{\EEQA}{\end{eqnarray}}
\newcommand{\define}{\stackrel{\triangle}{=}}
\bibliographystyle{IEEEtran}
\raggedbottom
\setlength{\parindent}{0pt}
\providecommand{\mbf}{\mathbf}
\providecommand{\pr}[1]{\ensuremath{\Pr\left(#1\right)}}
\providecommand{\qfunc}[1]{\ensuremath{Q\left(#1\right)}}
\providecommand{\sbrak}[1]{\ensuremath{{}\left[#1\right]}}
\providecommand{\lsbrak}[1]{\ensuremath{{}\left[#1\right.}}
\providecommand{\rsbrak}[1]{\ensuremath{{}\left.#1\right]}}
\providecommand{\brak}[1]{\ensuremath{\left(#1\right)}}
\providecommand{\lbrak}[1]{\ensuremath{\left(#1\right.}}
\providecommand{\rbrak}[1]{\ensuremath{\left.#1\right)}}
\providecommand{\cbrak}[1]{\ensuremath{\left\{#1\right\}}}
\providecommand{\lcbrak}[1]{\ensuremath{\left\{#1\right.}}
\providecommand{\rcbrak}[1]{\ensuremath{\left.#1\right\}}}
\theoremstyle{remark}
\newtheorem{rem}{Remark}
\newcommand{\sgn}{\mathop{\mathrm{sgn}}}
\providecommand{\abs}[1]{\left\vert#1\right\vert}
\providecommand{\res}[1]{\Res\displaylimits_{#1}} 
\providecommand{\norm}[1]{\left\lVert#1\right\rVert}
%\providecommand{\norm}[1]{\lVert#1\rVert}
\providecommand{\mtx}[1]{\mathbf{#1}}
\providecommand{\mean}[1]{E\left[ #1 \right]}
\providecommand{\fourier}{\overset{\mathcal{F}}{ \rightleftharpoons}}
%\providecommand{\hilbert}{\overset{\mathcal{H}}{ \rightleftharpoons}}
\providecommand{\system}{\overset{\mathcal{H}}{ \longleftrightarrow}}
	%\newcommand{\solution}[2]{\textbf{Solution:}{#1}}
\newcommand{\solution}{\noindent \textbf{Solution: }}
\newcommand{\cosec}{\,\text{cosec}\,}
\providecommand{\dec}[2]{\ensuremath{\overset{#1}{\underset{#2}{\gtrless}}}}
\newcommand{\myvec}[1]{\ensuremath{\begin{pmatrix}#1\end{pmatrix}}}
\newcommand{\mydet}[1]{\ensuremath{\begin{vmatrix}#1\end{vmatrix}}}
\numberwithin{equation}{subsection}
\makeatletter
\@addtoreset{figure}{problem}
\makeatother
\let\StandardTheFigure\thefigure
\let\vec\mathbf
\renewcommand{\thefigure}{\theproblem}
\def\putbox#1#2#3{\makebox[0in][l]{\makebox[#1][l]{}\raisebox{\baselineskip}[0in][0in]{\raisebox{#2}[0in][0in]{#3}}}}
     \def\rightbox#1{\makebox[0in][r]{#1}}
     \def\centbox#1{\makebox[0in]{#1}}
     \def\topbox#1{\raisebox{-\baselineskip}[0in][0in]{#1}}
     \def\midbox#1{\raisebox{-0.5\baselineskip}[0in][0in]{#1}}
\vspace{3cm}
\title{Assignment 16}
\author{Venkatesh E\\AI20MTECH14005}
\maketitle
\bigskip
\renewcommand{\thefigure}{\theenumi}
\renewcommand{\thetable}{\theenumi}
Download latex-tikz codes from 
\begin{lstlisting}
https://github.com/venkateshelangovan/IIT-Hyderabad-Assignments/tree/master/Assignment16_Matrix_Theory
\end{lstlisting}
\section{\textbf{Problem}}
If $\vec{P}$ and $\vec{Q}$ are invertible matrices such that   
$\vec{P}\vec{Q} = -\vec{Q}\vec{P}$,then we can conclude that
\begin{align}
1. Tr(\vec{P})&=Tr(\vec{Q})=0 \label{1}\\
2. Tr(\vec{P})&=Tr(\vec{Q})=1 \label{2}\\
3. Tr(\vec{P})&=-Tr(\vec{Q}) \label{3}\\
4. Tr(\vec{P}) &\neq Tr(\vec{Q}) \label{4}
\end{align}
\section{\textbf{Explanation with Proof}}
\renewcommand{\thetable}{1}
\begin{longtable}{|l|l|}
\hline
\multirow{3}{*}{} & \\
\text{Given} & 
\text{$\vec{P}$ and $\vec{Q}$ are invertible matrices}. \\
& \text{Therefore $\vec{P}^{-1}$ and $\vec{Q}^{-1}$ exists}.\\
& \parbox{10cm}{\begin{align}
    \vec{P}\vec{Q} =-\vec{Q}\vec{P} \label{given1}
\end{align}} 
\\ [0.5ex]
\hline
\text{To Prove} & \text{Tr($\vec{P}$)=0}\\
\hline
\text{Proof 1} & 
\text{Post multiplying equation \eqref{given1} by $\vec{Q}^{-1}$ we get,} \\
& \parbox{10cm}{\begin{align}
    \vec{P}\vec{Q}\vec{Q}^{-1}&=-\vec{Q}\vec{P}\vec{Q}^{-1}\\
    \implies \vec{P}\vec{I}&=-\vec{Q}\vec{P}\vec{Q}^{-1}\\
    \implies \vec{P}&=-\vec{Q}\vec{P}\vec{Q}^{-1}\label{p11}
\end{align}} \\
& Taking trace on both sides for the equation \eqref{p11}, \\
& \parbox{10cm}{\begin{align}
    Tr(\vec{P})&=Tr(-\vec{Q}\vec{P}\vec{Q}^{-1})\\
    \implies Tr(\vec{P})&=-Tr(\vec{Q}\vec{P}\vec{Q}^{-1})\label{p12}
\end{align}} \\
& \text{We know that Tr($\vec{A}\vec{B}$)=Tr($\vec{B}\vec{A}$)}
\\
& \text{Let $\vec{A}$=$\vec{Q}$ and $\vec{B}$=$\vec{P}\vec{Q}^{-1}$}\\
& \text{From the above property of trace equation \eqref{p12} can be modified as}\\
&\parbox{10cm}{\begin{align}
    Tr(\vec{P})&=-Tr(\vec{P}\vec{Q}^{-1}\vec{Q})\\
    \implies Tr(\vec{P})&=-Tr(\vec{P}\vec{I})
    \end{align}}\\
\hline
\pagebreak
\hline
\multirow{3}{*}&\\
&\parbox{10cm}{\begin{align}
    \implies Tr(\vec{P})&=-Tr(\vec{P})\label{c3}\\
    \implies 2Tr(\vec{P})&=0\\
    \implies Tr(\vec{P})&=0\label{pf}
\end{align}}\\
\hline
\text{To Prove} & \text{Tr($\vec{Q}$)=0}\\
\hline
\text{Proof 2} & 
\text{Post multiplying equation \eqref{given1} by $\vec{P}^{-1}$ we get,} \\
& \parbox{10cm}{\begin{align}
    \vec{P}\vec{Q}\vec{P}^{-1}&=-\vec{Q}\vec{P}\vec{P}^{-1}\\
    \implies \vec{P}\vec{Q}\vec{P}^{-1}&=-\vec{Q}\vec{I}\\
    \implies \vec{P}\vec{Q}\vec{P}^{-1}&=-\vec{Q}\label{p21}
\end{align}} \\
& Taking trace on both sides for the equation \eqref{p21}, \\
& \parbox{10cm}{\begin{align}
    Tr(\vec{P}\vec{Q}\vec{P}^{-1})&=Tr(-\vec{Q})\\
    \implies Tr(\vec{P}\vec{Q}\vec{P}^{-1})&=-Tr(\vec{Q})\label{p22}
\end{align}} \\
& \text{We know that Tr($\vec{A}\vec{B}$)=Tr($\vec{B}\vec{A}$)}
\\
& \text{Let $\vec{A}$=$\vec{P}$ and $\vec{B}$=$\vec{Q}\vec{P}^{-1}$}\\
& \text{From the above property of trace equation \eqref{p22} can be modified as}\\
&\parbox{10cm}{\begin{align}
    Tr(\vec{Q}\vec{P}^{-1}\vec{P})&=-Tr(\vec{Q})\\
    \implies Tr(\vec{Q}\vec{I})&=-Tr(\vec{Q})\\
    \implies Tr(\vec{Q})&=-Tr(\vec{Q})\\
    \implies 2Tr(\vec{Q})&=0\\
    \implies Tr(\vec{Q})&=0\label{qf}
\end{align}}\\
\hline
\text{\textbf{Statement 1}} & \text{Tr($\vec{P}$)=Tr($\vec{Q}$)=0}\\
\hline
\text{Explanation} & From equation \eqref{pf} and \eqref{qf} we could say that,
& &\parbox{10cm}{\begin{align}
     Tr(\vec{P})=Tr(\vec{Q})=0 \label{co1}
     \end{align}}\\
& \parbox{10cm}{\begin{center}
Valid Conclusion
\end{center}}
\hline
\text{\textbf{Statement 2}} & \text{Tr($\vec{P})=Tr(\vec{Q})=1$}\\
\hline
\text{Explanation} & From equation \eqref{pf} and \eqref{qf} we could say that,
& &\parbox{10cm}{\begin{align}
     Tr(\vec{P})=Tr(\vec{Q})\neq 1 
     \end{align}}\\
& \parbox{10cm}{\begin{center}
Invalid Conclusion
\end{center}}
\hline
\pagebreak
\hline
\multirow{3}{*}&\\
\text{\textbf{Statement 3}} & \text{Tr($\vec{P})=-Tr(\vec{Q})$}\\
\hline
\text{Explanation} & Substituting the conclusion 1 result equation \eqref{co1} in equation \eqref{c3} we get,
& &\parbox{10cm}{\begin{align}
     Tr(\vec{P})=-Tr(\vec{Q}) 
     \end{align}}\\
& \parbox{10cm}{\begin{center}
Valid Conclusion
\end{center}}
\hline
\text{\textbf{Statement 4}} & \text{Tr($\vec{P}) \neq Tr(\vec{Q})$}\\
\hline
\text{Explanation} & From equation \eqref{pf} and \eqref{qf} we could say that,
& &\parbox{10cm}{\begin{align}
     Tr(\vec{P})=Tr(\vec{Q}) 
     \end{align}}\\
& \parbox{10cm}{\begin{center}
Invalid Conclusion
\end{center}}
\hline
\caption{Explanation with Proofs}
\label{table:1}
\end{longtable}
\end{document}