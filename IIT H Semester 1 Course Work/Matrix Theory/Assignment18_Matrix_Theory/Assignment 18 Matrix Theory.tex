
\documentclass[journal,12pt]{IEEEtran}
\usepackage{longtable}
\usepackage{setspace}
\usepackage{gensymb}
\singlespacing
\usepackage[cmex10]{amsmath}
\newcommand\myemptypage{
\null
\thispagestyle{empty}
\addtocounter{page}{-1}
\newpage
}
\usepackage{amsthm}
\usepackage{mdframed}
\usepackage{mathrsfs}
\usepackage{txfonts}
\usepackage{stfloats}
\usepackage{bm}
\usepackage{cite}
\usepackage{cases}
\usepackage{subfig}

\usepackage{longtable}
\usepackage{multirow}

\usepackage{enumitem}
\usepackage{mathtools}
\usepackage{steinmetz}
\usepackage{tikz}
\usepackage{circuitikz}
\usepackage{verbatim}
\usepackage{tfrupee}
\usepackage[breaklinks=true]{hyperref}
\usepackage{graphicx}
\usepackage{tkz-euclide}

\usetikzlibrary{calc,math}
\usepackage{listings}
    \usepackage{color}                                            %%
    \usepackage{array}                                            %%
    \usepackage{longtable}                                        %%
    \usepackage{calc}                                             %%
    \usepackage{multirow}                                         %%
    \usepackage{hhline}                                           %%
    \usepackage{ifthen}                                           %%
    \usepackage{lscape}    
\usepackage{multicol}
\usepackage{chngcntr}

\DeclareMathOperator*{\Res}{Res}

\renewcommand\thesection{\arabic{section}}
\renewcommand\thesubsection{\thesection.\arabic{subsection}}
\renewcommand\thesubsubsection{\thesubsection.\arabic{subsubsection}}

\renewcommand\thesectiondis{\arabic{section}}
\renewcommand\thesubsectiondis{\thesectiondis.\arabic{subsection}}
\renewcommand\thesubsubsectiondis{\thesubsectiondis.\arabic{subsubsection}}


\hyphenation{op-tical net-works semi-conduc-tor}
\def\inputGnumericTable{}                                 %%

\lstset{
%language=C,
frame=single,
breaklines=true,
columns=fullflexible
}
\begin{document}
\onecolumn

\newtheorem{theorem}{Theorem}[section]
\newtheorem{problem}{Problem}
\newtheorem{proposition}{Proposition}[section]
\newtheorem{lemma}{Lemma}[section]
\newtheorem{corollary}[theorem]{Corollary}
\newtheorem{example}{Example}[section]
\newtheorem{definition}[problem]{Definition}

\newcommand{\BEQA}{\begin{eqnarray}}
\newcommand{\EEQA}{\end{eqnarray}}
\newcommand{\define}{\stackrel{\triangle}{=}}
\bibliographystyle{IEEEtran}
\raggedbottom
\setlength{\parindent}{0pt}
\providecommand{\mbf}{\mathbf}
\providecommand{\pr}[1]{\ensuremath{\Pr\left(#1\right)}}
\providecommand{\qfunc}[1]{\ensuremath{Q\left(#1\right)}}
\providecommand{\sbrak}[1]{\ensuremath{{}\left[#1\right]}}
\providecommand{\lsbrak}[1]{\ensuremath{{}\left[#1\right.}}
\providecommand{\rsbrak}[1]{\ensuremath{{}\left.#1\right]}}
\providecommand{\brak}[1]{\ensuremath{\left(#1\right)}}
\providecommand{\lbrak}[1]{\ensuremath{\left(#1\right.}}
\providecommand{\rbrak}[1]{\ensuremath{\left.#1\right)}}
\providecommand{\cbrak}[1]{\ensuremath{\left\{#1\right\}}}
\providecommand{\lcbrak}[1]{\ensuremath{\left\{#1\right.}}
\providecommand{\rcbrak}[1]{\ensuremath{\left.#1\right\}}}
\theoremstyle{remark}
\newtheorem{rem}{Remark}
\newcommand{\sgn}{\mathop{\mathrm{sgn}}}
\providecommand{\abs}[1]{\left\vert#1\right\vert}
\providecommand{\res}[1]{\Res\displaylimits_{#1}}
\providecommand{\norm}[1]{\left\lVert#1\right\rVert}
%\providecommand{\norm}[1]{\lVert#1\rVert}
\providecommand{\mtx}[1]{\mathbf{#1}}
\providecommand{\mean}[1]{E\left[ #1 \right]}
\providecommand{\fourier}{\overset{\mathcal{F}}{ \rightleftharpoons}}
%\providecommand{\hilbert}{\overset{\mathcal{H}}{ \rightleftharpoons}}
\providecommand{\system}{\overset{\mathcal{H}}{ \longleftrightarrow}}
%\newcommand{\solution}[2]{\textbf{Solution:}{#1}}
\newcommand{\solution}{\noindent \textbf{Solution: }}
\newcommand{\cosec}{\,\text{cosec}\,}
\providecommand{\dec}[2]{\ensuremath{\overset{#1}{\underset{#2}{\gtrless}}}}
\newcommand{\myvec}[1]{\ensuremath{\begin{pmatrix}#1\end{pmatrix}}}
\newcommand{\mydet}[1]{\ensuremath{\begin{vmatrix}#1\end{vmatrix}}}
\numberwithin{equation}{subsection}
\makeatletter
\@addtoreset{figure}{problem}
\makeatother
\let\StandardTheFigure\thefigure
\let\vec\mathbf
\renewcommand{\thefigure}{\theproblem}
\def\putbox#1#2#3{\makebox[0in][l]{\makebox[#1][l]{}\raisebox{\baselineskip}[0in][0in]{\raisebox{#2}[0in][0in]{#3}}}}
     \def\rightbox#1{\makebox[0in][r]{#1}}
     \def\centbox#1{\makebox[0in]{#1}}
     \def\topbox#1{\raisebox{-\baselineskip}[0in][0in]{#1}}
     \def\midbox#1{\raisebox{-0.5\baselineskip}[0in][0in]{#1}}
\vspace{3cm}
\title{Assignment 18}
\author{Venkatesh E\\AI20MTECH14005}
\maketitle
\bigskip
\renewcommand{\thefigure}{\theenumi}
\renewcommand{\thetable}{\theenumi}
Download latex-tikz codes from
\begin{lstlisting}
https://github.com/venkateshelangovan/IIT-Hyderabad-Assignments/tree/master/Assignment18_Matrix_Theory
\end{lstlisting}
\section{\textbf{Problem}}
If $\vec{A}$ be the $n \times n$ matrix(with $n>1$) satisfying $\vec{A}^2-7\vec{A}+12\vec{I}_{n\times n}=\vec{0}_{n\times n}$ where $\vec{I}_{n\times n}$ and $\vec{0}_{n\times n}$ denotes the identity matrix and zero matrix of order n respectively. Then which of the following statements are true ? 
\begin{enumerate}
\item $\vec{A}$ is invertible  
\item $t^2$-7t+12n=0 where t=Tr($\vec{A}$)
\item $d^2$-7d+12=0 where d=Det($\vec{A}$)
\item  $\lambda^2$-7$\lambda$+12=0 where $\lambda$ is an eigen value of $\vec{A}$
\end{enumerate}
\section{\textbf{Solution}}
\renewcommand{\thetable}{1}
\begin{longtable}{|l|l|}
\hline
\text{Given} &
\text{$\vec{A}$ be the $n\times n$ matrix where $n>1$ satisfying the following equation } \\
& \parbox{10cm}{\begin{align}
    \vec{A}^2-7\vec{A}+12\vec{I}_{n\times n}=\vec{0}_{n\times n} \label{given}
\end{align}}\\
\hline
\text{Explanation} & \text{The Cayley Hamilton Theorem states that every square matrix satisfies its own characteristic}\\
& \text{equation}.\\
& \text{Using this theorem the given equation \eqref{given} can be written as ,}\\
& \parbox{10cm}{\begin{align}
    \lambda^2-7\lambda+12&=0\label{4}\\
    (\lambda-4)(\lambda-3)&=0\\
    \lambda_1&=3\\
    \lambda_2&=4
\end{align}}\\
& \text{Here $\lambda_1$ and $\lambda_2$ were eigen values of matrix $\vec{A}$}\\
& \text{We know that determinant is product of eigen values.}\\
& \parbox{10cm}{\begin{align}
    d&=Det(\vec{A})\\
    \implies d &=\lambda_1\lambda_2\\
     \implies d&=12\neq 0 \label{1}
\end{align}}\\
\hline
\textbf{Statement 1} & \text{\vec{A} is invertible}
\hline
& \text{From equation \eqref{1}, since $d \neq 0$ the given matrix $\vec{A}$ is Invertible}.\\
& \parbox{10cm}{\begin{center}
\textbf{True Statement }
\end{center}}\\
\hline 
\textbf{Statement 2} & \parbox{10cm}{\begin{align}
    t^2-7t+12n=0 \label{trace}
\end{align}}\\
\hline
& We know that the trace is the sum of the eigen values.\\
& \parbox{10cm}{\begin{align}
    t&=Tr(\vec{A})\\
    \implies t &=\lambda_1+\lambda_2\\
     \implies t&=7 \label{t}
\end{align}}\\
& Substituting the equation \eqref{t} in \eqref{trace} we get,\\
& \parbox{10cm}{\begin{align}
    7^2-7(7)+12n&=0\\
    12n&=0\label{2}
\end{align}}\\
& Since given that $n>1$ the equation \eqref{2} is not possible i.e $12n\neq 0$.\\
& \parbox{10cm}{\begin{center}
Therefore, $t^2-7t+12n=0$ is a \textbf{False Statement }
\end{center}}\\
\hline 
\textbf{Statement 3} & \parbox{10cm}{\begin{align}
    d^2-7d+12=0 \label{det}
\end{align}}\\
\hline
& Substituting the equation \eqref{1} in \eqref{det}, we get, \\
& \parbox{10cm}{\begin{align}
    12^2-7(12)+12&=0\\
    72&=0 \label{2}
\end{align}} \\
& From equation \eqref{2} it is clear that the above statement 3 is invalid.\\
& \parbox{10cm}{\begin{center}
\textbf{False Statement} 
\end{center}}\\
\hline
\textbf{Statement 4} & \parbox{10cm}{\begin{align}
    \lambda^2-7\lambda+12=0 \label{eigen}
\end{align}}\\
\hline
& \text{By Cayley Hamilton Theorem, equation \eqref{4} shows that the above statement 4 is valid.}\\
& \parbox{10cm}{\begin{center}
\textbf{True Statement }
\end{center}}\\
\hline
\caption{Explanation}
\label{table:1}
\end{longtable}
\end{document}