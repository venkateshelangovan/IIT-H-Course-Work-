\documentclass[journal,12pt,twocolumn]{IEEEtran}
%
\usepackage{setspace}
\usepackage{gensymb}
\usepackage{siunitx}
\usepackage{tkz-euclide} 
\usepackage{textcomp}
\usepackage{standalone}
\usetikzlibrary{calc}

%\doublespacing
\singlespacing

%\usepackage{graphicx}
%\usepackage{amssymb}
%\usepackage{relsize}
\usepackage[cmex10]{amsmath}
%\usepackage{amsthm}
%\interdisplaylinepenalty=2500
%\savesymbol{iint}
%\usepackage{txfonts}
%\restoresymbol{TXF}{iint}
%\usepackage{wasysym}
\usepackage{amsthm}
%\usepackage{iithtlc}
\usepackage{mathrsfs}
\usepackage{txfonts}
\usepackage{stfloats}
\usepackage{bm}
\usepackage{cite}
\usepackage{cases}
\usepackage{subfig}
%\usepackage{xtab}
\usepackage{longtable}
\usepackage{multirow}
%\usepackage{algorithm}
%\usepackage{algpseudocode}
\usepackage{enumitem}
\usepackage{mathtools}
\usepackage{steinmetz}
\usepackage{tikz}
\usepackage{circuitikz}
\usepackage{verbatim}
\usepackage{tfrupee}
\usepackage[breaklinks=true]{hyperref}
%\usepackage{stmaryrd}
\usepackage{tkz-euclide} % loads  TikZ and tkz-base
%\usetkzobj{all}
\usetikzlibrary{calc,math}
\usepackage{listings}
    \usepackage{color}                                            %%
    \usepackage{array}                                            %%
    \usepackage{longtable}                                        %%
    \usepackage{calc}                                             %%
    \usepackage{multirow}                                         %%
    \usepackage{hhline}                                           %%
    \usepackage{ifthen}                                           %%
  %optionally (for landscape tables embedded in another document): %%
    \usepackage{lscape}     
\usepackage{multicol}
\usepackage{chngcntr}
\usepackage{amsmath}
\usepackage{cleveref}
%\usepackage{enumerate}

%\usepackage{wasysym}
%\newcounter{MYtempeqncnt}
\DeclareMathOperator*{\Res}{Res}
%\renewcommand{\baselinestretch}{2}
\renewcommand\thesection{\arabic{section}}
\renewcommand\thesubsection{\thesection.\arabic{subsection}}
\renewcommand\thesubsubsection{\thesubsection.\arabic{subsubsection}}

\renewcommand\thesectiondis{\arabic{section}}
\renewcommand\thesubsectiondis{\thesectiondis.\arabic{subsection}}
\renewcommand\thesubsubsectiondis{\thesubsectiondis.\arabic{subsubsection}}

% correct bad hyphenation here
\hyphenation{op-tical net-works semi-conduc-tor}
\def\inputGnumericTable{}                                 %%

\lstset{
%language=C,
frame=single, 
breaklines=true,
columns=fullflexible
}
%\lstset{
%language=tex,
%frame=single, 
%breaklines=true
%}
\usepackage{graphicx}
\usepackage{pgfplots}

\begin{document}
%


\newtheorem{theorem}{Theorem}[section]
\newtheorem{problem}{Problem}
\newtheorem{proposition}{Proposition}[section]
\newtheorem{lemma}{Lemma}[section]
\newtheorem{corollary}[theorem]{Corollary}
\newtheorem{example}{Example}[section]
\newtheorem{definition}[problem]{Definition}
%\newtheorem{thm}{Theorem}[section] 
%\newtheorem{defn}[thm]{Definition}
%\newtheorem{algorithm}{Algorithm}[section]
%\newtheorem{cor}{Corollary}
\newcommand{\BEQA}{\begin{eqnarray}}
\newcommand{\EEQA}{\end{eqnarray}}
\newcommand{\define}{\stackrel{\triangle}{=}}
\bibliographystyle{IEEEtran}
%\bibliographystyle{ieeetr}
\providecommand{\mbf}{\mathbf}
\providecommand{\pr}[1]{\ensuremath{\Pr\left(#1\right)}}
\providecommand{\qfunc}[1]{\ensuremath{Q\left(#1\right)}}
\providecommand{\sbrak}[1]{\ensuremath{{}\left[#1\right]}}
\providecommand{\lsbrak}[1]{\ensuremath{{}\left[#1\right.}}
\providecommand{\rsbrak}[1]{\ensuremath{{}\left.#1\right]}}
\providecommand{\brak}[1]{\ensuremath{\left(#1\right)}}
\providecommand{\lbrak}[1]{\ensuremath{\left(#1\right.}}
\providecommand{\rbrak}[1]{\ensuremath{\left.#1\right)}}
\providecommand{\cbrak}[1]{\ensuremath{\left\{#1\right\}}}
\providecommand{\lcbrak}[1]{\ensuremath{\left\{#1\right.}}
\providecommand{\rcbrak}[1]{\ensuremath{\left.#1\right\}}}
\theoremstyle{remark}
\newtheorem{rem}{Remark}
\newcommand{\sgn}{\mathop{\mathrm{sgn}}}
\providecommand{\abs}[1]{\left\vert#1\right\vert}
\providecommand{\res}[1]{\Res\displaylimits_{#1}} 
\providecommand{\norm}[1]{\left\lVert#1\right\rVert}
%\providecommand{\norm}[1]{\lVert#1\rVert}
\providecommand{\mtx}[1]{\mathbf{#1}}
\providecommand{\mean}[1]{E\left[ #1 \right]}
\providecommand{\fourier}{\overset{\mathcal{F}}{ \rightleftharpoons}}
%\providecommand{\hilbert}{\overset{\mathcal{H}}{ \rightleftharpoons}}
\providecommand{\system}{\overset{\mathcal{H}}{ \longleftrightarrow}}
	%\newcommand{\solution}[2]{\textbf{Solution:}{#1}}
\newcommand{\solution}{\noindent \textbf{Solution: }}
\newcommand{\cosec}{\,\text{cosec}\,}
\providecommand{\dec}[2]{\ensuremath{\overset{#1}{\underset{#2}{\gtrless}}}}
\newcommand{\myvec}[1]{\ensuremath{\begin{pmatrix}#1\end{pmatrix}}}
\newcommand{\mydet}[1]{\ensuremath{\begin{vmatrix}#1\end{vmatrix}}}
%\numberwithin{equation}{section}
\numberwithin{equation}{subsection}
%\numberwithin{problem}{section}
%\numberwithin{definition}{section}
\makeatletter
\@addtoreset{figure}{problem}
\makeatother
\let\StandardTheFigure\thefigure
\let\vec\mathbf
%\renewcommand{\thefigure}{\theproblem.\arabic{figure}}
\renewcommand{\thefigure}{\theproblem}
%\setlist[enumerate,1]{before=\renewcommand\theequation{\theenumi.\arabic{equation}}
%\counterwithin{equation}{enumi}
%\renewcommand{\theequation}{\arabic{subsection}.\arabic{equation}}
\def\putbox#1#2#3{\makebox[0in][l]{\makebox[#1][l]{}\raisebox{\baselineskip}[0in][0in]{\raisebox{#2}[0in][0in]{#3}}}}
     \def\rightbox#1{\makebox[0in][r]{#1}}
     \def\centbox#1{\makebox[0in]{#1}}
     \def\topbox#1{\raisebox{-\baselineskip}[0in][0in]{#1}}
     \def\midbox#1{\raisebox{-0.5\baselineskip}[0in][0in]{#1}}
\vspace{3cm}
\title{Assignment 9}
\author{Venkatesh E\\AI20MTECH14005}
\maketitle
\newpage
%\tableofcontents
\bigskip
\renewcommand{\thefigure}{\theenumi}
\renewcommand{\thetable}{\theenumi}
\begin{abstract}
This document performs QR decomposition on a given 2X2  matrix.
\end{abstract}
Download all latex-tikz codes from 
%
\begin{lstlisting}
https://github.com/venkateshelangovan/IIT-Hyderabad-Assignments/tree/master/Assignment9_Matrix_Theory
\end{lstlisting}
\section{Problem}
Find QR decomposition of \myvec{4&3\\5&-2}
\section{QR Decomposition}
The QR decomposition  of a matrix is a decomposition of the matrix into an orthogonal matrix and an upper triangular matrix. A QR decomposition of a real square matrix A is a decomposition of A as
\begin{align}
    \vec{A}&=\vec{Q}\vec{R}
\end{align}
where $\vec{Q}$ is an orthogonal matrix and $\vec{R}$ is an upper triangular matrix
\section{Solution}
Given 
\begin{align}
    \vec{A}=\myvec{4&3\\5&-2}
\end{align}
Let $\vec{a}$ and $\vec{b}$ be the column vectors of the given matrix
\begin{align}
\vec{a}&=\myvec{4\\5}\label{a}\\
\vec{b}&=\myvec{3\\-2}\label{b}
\end{align}
The above column vectors \eqref{a} ,\eqref{b} can be expressed as ,
\begin{align}
\vec{a}&=t_1\vec{u}_1\label{a1}\\
\vec{b}&=s_1\vec{u}_1+t_2\vec{u}_2\label{a2}
\end{align}
Where, 
\begin{align}
t_1 &= \norm{\vec{a}}\label{start}\\
\vec{u}_1 &= \frac{\vec{a}}{t_1}\\
s_1 &= \frac{\vec{u}_1^T\vec{b}}{\norm{\vec{u}_1}^2}\\
\vec{u}_2 &= \frac{\vec{b} - s_1 \vec{u}_1}{\norm{\vec{b} - s_1 \vec{u}_1}}\\
t_2 &= {\vec{u}_2^T\vec{b}}\label{end}
\end{align}
The \eqref{a1} and \eqref{a2} can be written as, 
\begin{align}
\myvec{\vec{a} & \vec{b}} &= \myvec{\vec{u}_1 & \vec{u}_2}\myvec{t_1 & s_1 \\ 0 & t_2}\\
\myvec{\vec{a} & \vec{b}} &= \vec{Q}\vec{R}\label{final}
\end{align}
Here, $\vec{R}$ is an upper triangular matrix and $\vec{Q}$ is an orthogonal matrix such that
\begin{align}
\vec{Q}^T\vec{Q}=\vec{I}
\end{align}
Now using equations from \eqref{start} to \eqref{end} we get, 
\begin{align}
t_1 &= \sqrt{4^2+5^2} = \sqrt{41}\label{t1}\\ 
\vec{u}_1 &= \frac{1}{\sqrt{41}}\myvec{4\\5}\\
s_1 &= \myvec{\frac{4}{\sqrt{41}}&\frac{5}{\sqrt{41}}}\myvec{3\\-2} = \frac{2}{\sqrt{41}}\\ 
\vec{u}_2 &= \frac{1}{\sqrt{41}}\myvec{5\\-4}\\
t_2 &= \myvec{\frac{5}{\sqrt{41}}&\frac{-4}{\sqrt{41}}}\myvec{3\\-2} = \frac{23}{\sqrt{41}}\label{t2} 
\end{align}
Substituting the values from \eqref{t1} to \eqref{t2} in \eqref{final} we obtain QR decomposition as,
\begin{align}
\myvec{4&3\\5&-2}=\myvec{\frac{4}{\sqrt{41}}&\frac{5}{\sqrt{41}}\\\frac{5}{\sqrt{41}}&\frac{-4}{\sqrt{41}}}\myvec{\sqrt{41}&\frac{2}{\sqrt{41}}\\0&\frac{23}{\sqrt{41}}}
\end{align}
\end{document}