\documentclass[journal,12pt,twocolumn]{IEEEtran}
%
\usepackage{setspace}
\usepackage{gensymb}
\usepackage{siunitx}
\usepackage{tkz-euclide} 
\usepackage{textcomp}
\usepackage{standalone}
\usetikzlibrary{calc}

%\doublespacing
\singlespacing

%\usepackage{graphicx}
%\usepackage{amssymb}
%\usepackage{relsize}
\usepackage[cmex10]{amsmath}
%\usepackage{amsthm}
%\interdisplaylinepenalty=2500
%\savesymbol{iint}
%\usepackage{txfonts}
%\restoresymbol{TXF}{iint}
%\usepackage{wasysym}
\usepackage{amsthm}
%\usepackage{iithtlc}
\usepackage{mathrsfs}
\usepackage{txfonts}
\usepackage{stfloats}
\usepackage{bm}
\usepackage{cite}
\usepackage{cases}
\usepackage{subfig}
%\usepackage{xtab}
\usepackage{longtable}
\usepackage{multirow}
%\usepackage{algorithm}
%\usepackage{algpseudocode}
\usepackage{enumitem}
\usepackage{mathtools}
\usepackage{steinmetz}
\usepackage{tikz}
\usepackage{circuitikz}
\usepackage{verbatim}
\usepackage{tfrupee}
\usepackage[breaklinks=true]{hyperref}
%\usepackage{stmaryrd}
\usepackage{tkz-euclide} % loads  TikZ and tkz-base
%\usetkzobj{all}
\usetikzlibrary{calc,math}
\usepackage{listings}
    \usepackage{color}                                            %%
    \usepackage{array}                                            %%
    \usepackage{longtable}                                        %%
    \usepackage{calc}                                             %%
    \usepackage{multirow}                                         %%
    \usepackage{hhline}                                           %%
    \usepackage{ifthen}                                           %%
  %optionally (for landscape tables embedded in another document): %%
    \usepackage{lscape}     
\usepackage{multicol}
\usepackage{chngcntr}
\usepackage{amsmath}
\usepackage{cleveref}
%\usepackage{enumerate}

%\usepackage{wasysym}
%\newcounter{MYtempeqncnt}
\DeclareMathOperator*{\Res}{Res}
%\renewcommand{\baselinestretch}{2}
\renewcommand\thesection{\arabic{section}}
\renewcommand\thesubsection{\thesection.\arabic{subsection}}
\renewcommand\thesubsubsection{\thesubsection.\arabic{subsubsection}}

\renewcommand\thesectiondis{\arabic{section}}
\renewcommand\thesubsectiondis{\thesectiondis.\arabic{subsection}}
\renewcommand\thesubsubsectiondis{\thesubsectiondis.\arabic{subsubsection}}

% correct bad hyphenation here
\hyphenation{op-tical net-works semi-conduc-tor}
\def\inputGnumericTable{}                                 %%

\lstset{
%language=C,
frame=single, 
breaklines=true,
columns=fullflexible
}
%\lstset{
%language=tex,
%frame=single, 
%breaklines=true
%}
\usepackage{graphicx}
\usepackage{pgfplots}

\begin{document}
%


\newtheorem{theorem}{Theorem}[section]
\newtheorem{problem}{Problem}
\newtheorem{proposition}{Proposition}[section]
\newtheorem{lemma}{Lemma}[section]
\newtheorem{corollary}[theorem]{Corollary}
\newtheorem{example}{Example}[section]
\newtheorem{definition}[problem]{Definition}
%\newtheorem{thm}{Theorem}[section] 
%\newtheorem{defn}[thm]{Definition}
%\newtheorem{algorithm}{Algorithm}[section]
%\newtheorem{cor}{Corollary}
\newcommand{\BEQA}{\begin{eqnarray}}
\newcommand{\EEQA}{\end{eqnarray}}
\newcommand{\define}{\stackrel{\triangle}{=}}
\bibliographystyle{IEEEtran}
%\bibliographystyle{ieeetr}
\providecommand{\mbf}{\mathbf}
\providecommand{\pr}[1]{\ensuremath{\Pr\left(#1\right)}}
\providecommand{\qfunc}[1]{\ensuremath{Q\left(#1\right)}}
\providecommand{\sbrak}[1]{\ensuremath{{}\left[#1\right]}}
\providecommand{\lsbrak}[1]{\ensuremath{{}\left[#1\right.}}
\providecommand{\rsbrak}[1]{\ensuremath{{}\left.#1\right]}}
\providecommand{\brak}[1]{\ensuremath{\left(#1\right)}}
\providecommand{\lbrak}[1]{\ensuremath{\left(#1\right.}}
\providecommand{\rbrak}[1]{\ensuremath{\left.#1\right)}}
\providecommand{\cbrak}[1]{\ensuremath{\left\{#1\right\}}}
\providecommand{\lcbrak}[1]{\ensuremath{\left\{#1\right.}}
\providecommand{\rcbrak}[1]{\ensuremath{\left.#1\right\}}}
\theoremstyle{remark}
\newtheorem{rem}{Remark}
\newcommand{\sgn}{\mathop{\mathrm{sgn}}}
\providecommand{\abs}[1]{\left\vert#1\right\vert}
\providecommand{\res}[1]{\Res\displaylimits_{#1}} 
\providecommand{\norm}[1]{\left\lVert#1\right\rVert}
%\providecommand{\norm}[1]{\lVert#1\rVert}
\providecommand{\mtx}[1]{\mathbf{#1}}
\providecommand{\mean}[1]{E\left[ #1 \right]}
\providecommand{\fourier}{\overset{\mathcal{F}}{ \rightleftharpoons}}
%\providecommand{\hilbert}{\overset{\mathcal{H}}{ \rightleftharpoons}}
\providecommand{\system}{\overset{\mathcal{H}}{ \longleftrightarrow}}
	%\newcommand{\solution}[2]{\textbf{Solution:}{#1}}
\newcommand{\solution}{\noindent \textbf{Solution: }}
\newcommand{\cosec}{\,\text{cosec}\,}
\providecommand{\dec}[2]{\ensuremath{\overset{#1}{\underset{#2}{\gtrless}}}}
\newcommand{\myvec}[1]{\ensuremath{\begin{pmatrix}#1\end{pmatrix}}}
\newcommand{\mydet}[1]{\ensuremath{\begin{vmatrix}#1\end{vmatrix}}}
%\numberwithin{equation}{section}
\numberwithin{equation}{subsection}
%\numberwithin{problem}{section}
%\numberwithin{definition}{section}
\makeatletter
\@addtoreset{figure}{problem}
\makeatother
\let\StandardTheFigure\thefigure
\let\vec\mathbf
%\renewcommand{\thefigure}{\theproblem.\arabic{figure}}
\renewcommand{\thefigure}{\theproblem}
%\setlist[enumerate,1]{before=\renewcommand\theequation{\theenumi.\arabic{equation}}
%\counterwithin{equation}{enumi}
%\renewcommand{\theequation}{\arabic{subsection}.\arabic{equation}}
\def\putbox#1#2#3{\makebox[0in][l]{\makebox[#1][l]{}\raisebox{\baselineskip}[0in][0in]{\raisebox{#2}[0in][0in]{#3}}}}
     \def\rightbox#1{\makebox[0in][r]{#1}}
     \def\centbox#1{\makebox[0in]{#1}}
     \def\topbox#1{\raisebox{-\baselineskip}[0in][0in]{#1}}
     \def\midbox#1{\raisebox{-0.5\baselineskip}[0in][0in]{#1}}
\vspace{3cm}
\title{Assignment 15}
\author{Venkatesh E\\AI20MTECH14005}
\maketitle
\newpage
%\tableofcontents
\bigskip
\renewcommand{\thefigure}{\theenumi}
\renewcommand{\thetable}{\theenumi}
\begin{abstract}
This document explains the concept of linear transformation from $\mathbb{R}^3$ into $\mathbb{R}^2$
\end{abstract}
Download all latex-tikz codes from 
%
\begin{lstlisting}
https://github.com/venkateshelangovan/IIT-Hyderabad-Assignments/tree/master/Assignment15_Matrix_Theory
\end{lstlisting}
\section{Problem}
 Is there a linear transformation $\vec{T}$ from $\mathbb{R}^3$ into $\mathbb{R}^2$ such that,
 \begin{align}
 \vec{T}\myvec{1\\ -1\\ 1} & = \myvec{1\\0}\label{g1}\\
 \vec{T}\myvec{1\\ 1\\1} &= \myvec{0\\ 1}\label{g2}
 \end{align}
 \section{Linear Transformation}
 A linear transformation is a function $\vec{T} :\mathbb{R}^n \rightarrow  \mathbb{R}^m$ which satisfies:
 
 1. $\forall \vec{x},\vec{y} \in \mathbb{R}^n$,
 \begin{align}
     \vec{T}\myvec{\vec{x}+\vec{y}}&=\vec{T}\myvec{\vec{x}}+\vec{T}\myvec{\vec{y}} 
 \end{align}
 2. $\forall \vec{x}\in \mathbb{R}^n$ and c $\in \mathbb{R}$,
 \begin{align}
     \vec{T}\myvec{c\vec{x}}=c\vec{T}\myvec{\vec{x}}
 \end{align}
\subsection{Matrix of the Linear Transformation}
Let,
$\vec{T} :\mathbb{R}^n \rightarrow  \mathbb{R}^m$ is a linear transformation and $\vec{x} \in \mathbb{R}^n$ is given by ,
\begin{align}
    \vec{x}&=\myvec{x_1\\x_2\\\vdots\\x_n}\\
    &=x_1\myvec{1\\0\\\vdots\\0}+x_2\myvec{0\\1\\\vdots\\0}+\dots+x_n\myvec{0\\0\\\vdots\\1}\label{1}
\end{align}
Let $\vec{e_1},\vec{e_2},\dots,\vec{e_n}$ be the standard basis of $\mathbb{R}^n$ and the equation \eqref{1} can be rewritten as,
\begin{align}
    \vec{x}=\sum_{i=1}^{n} x_i\vec{e_i}
\end{align}
\begin{align}
    \vec{T}\myvec{\vec{x}}&=\sum_{i=1}^{n} x_i\vec{T}\myvec{\vec{e_i}}\\
    &=\myvec{\vec{T}\myvec{\vec{e_1}} &\vec{T}\myvec{\vec{e_2}}& \dots &\vec{T}\myvec{\vec{e_n}}}\myvec{x_1\\x_2\\\vdots\\x_n}\\
   \vec{T}\myvec{\vec{x}} &=\vec{A}\vec{x}\label{ax}
\end{align}
Where,
\begin{align}
    \vec{A}=\myvec{\vec{T}\myvec{\vec{e_1}} &\vec{T}\myvec{\vec{e_n}}& \dots &\vec{T}\myvec{\vec{e_n}}}
\end{align}
If  $\vec{T}$  is any linear transformation which maps $\mathbb{R}^n \rightarrow  \mathbb{R}^m$ there is always an $m\times n$  matrix  $\vec{A}$  with the property that
\begin{align}
    \vec{T}\myvec{\vec{x}} &=\vec{A}\vec{x}\notag
\end{align}
where , $\vec{x} \in \mathbb{R}^n$
\section{Solution}
Let,
\begin{align}
    \vec{v}=\myvec{1\\-1\\1}\\
    \vec{u}=\myvec{1\\1\\1}
\end{align}
Given,
\begin{align}
    \vec{T}\myvec{\vec{v}}=\myvec{1\\0}\\
    \vec{T}\myvec{\vec{u}}=\myvec{0\\1}
\end{align}
Let the standard basis vectors is denoted as, 
\begin{align}
    \vec{e_1}&=\myvec{1\\0\\0}\\
    \vec{e_2}&=\myvec{0\\1\\0}\\
    \vec{e_3}&=\myvec{0\\0\\1}
\end{align}
Let, 
$\vec{T} :\mathbb{R}^3 \rightarrow  \mathbb{R}^2$ be a linear transformation. Then the function $\vec{T}$ is just matrix-vector multiplication $\vec{T}(\vec{x}) = \vec{A}\vec{x}$ for some matrix $\vec{A}$ as shown in equation \eqref{ax}

Matrix $\vec{A}$ of order $2 \times 3$ is given by,
\begin{align}
    \vec{A}=\myvec{\vec{T}\myvec{\vec{e_1}} & \vec{T}\myvec{\vec{e_2}} & \vec{T}\myvec{\vec{e_3}}}\label{A}
\end{align}
Consider the vector $\vec{b} \in \mathbb{R}^3$ which is the linear combinations of the vectors $\vec{v}$ and $\vec{u}$.

For $x_1,x_2 \in \mathbb{R}$,
\begin{align}
    \vec{b}=\myvec{b_1\\b_2\\b_3}=x_1\myvec{1\\-1\\1}+x_2\myvec{1\\1\\1}
\end{align}
\begin{align}
    \vec{T}\myvec{b_1\\b_2\\b_3}=x_1\vec{T}\myvec{1\\-1\\1}+x_2\vec{T}\myvec{1\\1\\1}\label{eq1}
\end{align}
To find $x_1,x_2$, we solve the linear system,$\vec{M}\vec{x}=\vec{b}$ where $\vec{M}$ is the $3 \times 2$ matrix obtained by stacking the given vectors $\vec{v}$ and $\vec{u}$ as columns
\begin{align}
    \vec{M}=\myvec{1 & 1\\-1 & 1 \\ 1 & 1}
\end{align}
\begin{align}
\myvec{1 & 1\\-1 & 1 \\ 1 & 1}\myvec{x_1\\x_2}=\myvec{b_1\\b_2\\b_3}\label{eq}
\end{align}
The augumented matrix of the equation \eqref{eq} is given by ,
\begin{align}
    &\myvec{1& 1 &\vrule & b_1 \\ -1 & 1 & \vrule& b_2\\ 1 & 1 &\vrule & b_3}\label{eqaug}
\end{align}
By row reducing the above equation \eqref{eqaug},
\begin{align}
    &\myvec{1& 1 &\vrule & b_1 \\ -1 & 1 & \vrule& b_2\\ 1 & 1 &\vrule & b_3}&\xleftrightarrow{R_2=R_2+R_1}&\myvec{1& 1 &\vrule & b_1 \\ 0 & 2 & \vrule& b_2+b_1\\ 1 & 1 &\vrule & b_3}
\end{align}
\begin{align}
&\myvec{1& 1 &\vrule & b_1 \\ 0 & 2 & \vrule& b_2+b_1\\ 1 & 1 &\vrule & b_3}&\xleftrightarrow{R_3=R_3-R_1}&\myvec{1& 1 &\vrule & b_1 \\ 0 & 2 & \vrule& b_2+b_1\\ 0 & 0 &\vrule & b_3-b_1}\\
&\myvec{1& 1 &\vrule & b_1 \\ 0 & 2 & \vrule& b_2+b_1\\ 0 & 0 &\vrule & b_3-b_1}&\xleftrightarrow{R_2=\frac{R_2}{2}}&\myvec{1& 1 &\vrule & b_1 \\ 0 & 1 & \vrule& \frac{b_2+b_1}{2}\\ 0 & 0 &\vrule & b_3-b_1}\\
&\myvec{1& 1 &\vrule & b_1 \\ 0 & 1 & \vrule& \frac{b_2+b_1}{2}\\ 0 & 0 &\vrule & b_3-b_1}&\xleftrightarrow{R_1=R_1-R_2}&\myvec{1& 0 &\vrule & \frac{b_1-b_2}{2} \\ 0 & 1 & \vrule& \frac{b_2+b_1}{2}\\ 0 & 0 &\vrule & b_3-b_1}
\end{align}
Now equation \eqref{eq} can be written as,
\begin{align}
    \myvec{1 & 0\\0 & 1 \\ 0& 0}\myvec{x_1\\x_2}=\myvec{\frac{b_1-b_2}{2}\\\frac{b_2+b_1}{2}\\b_3-b_1}
\end{align}
Solving the above equation we get ,
\begin{align}
    x_1&=\frac{b_1-b_2}{2}\label{x1}\\
    x_2&=\frac{b_1+b_2}{2}\label{x2}
\end{align}
Substituting the above equations \eqref{x1},\eqref{x2} in equation \eqref{eq1}, we get,
\begin{align}
    \vec{T}\myvec{b_1\\b_2\\b_3}&=\left(\frac{b_1-b_2}{2}\right)\vec{T}\myvec{1\\-1\\1}+\left(\frac{b_1+b_2}{2}\right)\vec{T}\myvec{1\\1\\1}\label{in}
\end{align}
Substituting the equations \eqref{g1} and \eqref{g2} in equation \eqref{in} we get,
\begin{align}
        \vec{T}\myvec{b_1\\b_2\\b_3}&=\left(\frac{b_1-b_2}{2}\right)\myvec{1\\0}+\left(\frac{b_1+b_2}{2}\right)\myvec{0\\1}\\
        \vec{T}\myvec{b_1\\b_2\\b_3}&=\myvec{\frac{b_1-b_2}{2}\\\frac{b_1+b_2}{2}}\label{ch}
\end{align}
Using the above equation \eqref{ch} we compute,
\begin{align}
    \vec{T}\myvec{\vec{e_1}}=\vec{T}\myvec{1\\0\\0}&=\myvec{\frac{1}{2}\\\frac{1}{2}}\label{a1}\\
    \vec{T}\myvec{\vec{e_2}}=\vec{T}\myvec{0\\1\\0}&=\myvec{\frac{-1}{2}\\\frac{1}{2}}\label{a2}\\
    \vec{T}\myvec{\vec{e_3}}=\vec{T}\myvec{0\\0\\1}&=\myvec{0\\0}\label{a3}
\end{align}
Substituting the equations \eqref{a1},\eqref{a2} and \eqref{a3} in equation \eqref{A} we get,
\begin{align}
    \vec{A}&=\myvec{\frac{1}{2} & \frac{-1}{2} & 0\\\frac{1}{2} & \frac{1}{2} & 0}\\
   \implies \vec{A}&=\frac{1}{2}\myvec{1 & -1 & 0\\1 & 1 & 0}
\end{align}
Therefore from the above matrix $\vec{A}$ we can say that there is a linear transformation $\vec{T}$ from $\mathbb{R}^3$ into $\mathbb{R}^2$ which satisfies the given conditions \eqref{g1} and \eqref{g2}.
\end{document}
