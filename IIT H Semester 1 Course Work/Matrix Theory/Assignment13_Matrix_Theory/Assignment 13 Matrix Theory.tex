\documentclass[journal,12pt,twocolumn]{IEEEtran}
%
\usepackage{setspace}
\usepackage{gensymb}
\usepackage{siunitx}
\usepackage{tkz-euclide} 
\usepackage{textcomp}
\usepackage{standalone}
\usetikzlibrary{calc}

%\doublespacing
\singlespacing

%\usepackage{graphicx}
%\usepackage{amssymb}
%\usepackage{relsize}
\usepackage[cmex10]{amsmath}
%\usepackage{amsthm}
%\interdisplaylinepenalty=2500
%\savesymbol{iint}
%\usepackage{txfonts}
%\restoresymbol{TXF}{iint}
%\usepackage{wasysym}
\usepackage{amsthm}
%\usepackage{iithtlc}
\usepackage{mathrsfs}
\usepackage{txfonts}
\usepackage{stfloats}
\usepackage{bm}
\usepackage{cite}
\usepackage{cases}
\usepackage{subfig}
%\usepackage{xtab}
\usepackage{longtable}
\usepackage{multirow}
%\usepackage{algorithm}
%\usepackage{algpseudocode}
\usepackage{enumitem}
\usepackage{mathtools}
\usepackage{steinmetz}
\usepackage{tikz}
\usepackage{circuitikz}
\usepackage{verbatim}
\usepackage{tfrupee}
\usepackage[breaklinks=true]{hyperref}
%\usepackage{stmaryrd}
\usepackage{tkz-euclide} % loads  TikZ and tkz-base
%\usetkzobj{all}
\usetikzlibrary{calc,math}
\usepackage{listings}
    \usepackage{color}                                            %%
    \usepackage{array}                                            %%
    \usepackage{longtable}                                        %%
    \usepackage{calc}                                             %%
    \usepackage{multirow}                                         %%
    \usepackage{hhline}                                           %%
    \usepackage{ifthen}                                           %%
  %optionally (for landscape tables embedded in another document): %%
    \usepackage{lscape}     
\usepackage{multicol}
\usepackage{chngcntr}
\usepackage{amsmath}
\usepackage{cleveref}
%\usepackage{enumerate}

%\usepackage{wasysym}
%\newcounter{MYtempeqncnt}
\DeclareMathOperator*{\Res}{Res}
%\renewcommand{\baselinestretch}{2}
\renewcommand\thesection{\arabic{section}}
\renewcommand\thesubsection{\thesection.\arabic{subsection}}
\renewcommand\thesubsubsection{\thesubsection.\arabic{subsubsection}}

\renewcommand\thesectiondis{\arabic{section}}
\renewcommand\thesubsectiondis{\thesectiondis.\arabic{subsection}}
\renewcommand\thesubsubsectiondis{\thesubsectiondis.\arabic{subsubsection}}

% correct bad hyphenation here
\hyphenation{op-tical net-works semi-conduc-tor}
\def\inputGnumericTable{}                                 %%

\lstset{
%language=C,
frame=single, 
breaklines=true,
columns=fullflexible
}
%\lstset{
%language=tex,
%frame=single, 
%breaklines=true
%}
\usepackage{graphicx}
\usepackage{pgfplots}

\begin{document}
%


\newtheorem{theorem}{Theorem}[section]
\newtheorem{problem}{Problem}
\newtheorem{proposition}{Proposition}[section]
\newtheorem{lemma}{Lemma}[section]
\newtheorem{corollary}[theorem]{Corollary}
\newtheorem{example}{Example}[section]
\newtheorem{definition}[problem]{Definition}
%\newtheorem{thm}{Theorem}[section] 
%\newtheorem{defn}[thm]{Definition}
%\newtheorem{algorithm}{Algorithm}[section]
%\newtheorem{cor}{Corollary}
\newcommand{\BEQA}{\begin{eqnarray}}
\newcommand{\EEQA}{\end{eqnarray}}
\newcommand{\define}{\stackrel{\triangle}{=}}
\bibliographystyle{IEEEtran}
%\bibliographystyle{ieeetr}
\providecommand{\mbf}{\mathbf}
\providecommand{\pr}[1]{\ensuremath{\Pr\left(#1\right)}}
\providecommand{\qfunc}[1]{\ensuremath{Q\left(#1\right)}}
\providecommand{\sbrak}[1]{\ensuremath{{}\left[#1\right]}}
\providecommand{\lsbrak}[1]{\ensuremath{{}\left[#1\right.}}
\providecommand{\rsbrak}[1]{\ensuremath{{}\left.#1\right]}}
\providecommand{\brak}[1]{\ensuremath{\left(#1\right)}}
\providecommand{\lbrak}[1]{\ensuremath{\left(#1\right.}}
\providecommand{\rbrak}[1]{\ensuremath{\left.#1\right)}}
\providecommand{\cbrak}[1]{\ensuremath{\left\{#1\right\}}}
\providecommand{\lcbrak}[1]{\ensuremath{\left\{#1\right.}}
\providecommand{\rcbrak}[1]{\ensuremath{\left.#1\right\}}}
\theoremstyle{remark}
\newtheorem{rem}{Remark}
\newcommand{\sgn}{\mathop{\mathrm{sgn}}}
\providecommand{\abs}[1]{\left\vert#1\right\vert}
\providecommand{\res}[1]{\Res\displaylimits_{#1}} 
\providecommand{\norm}[1]{\left\lVert#1\right\rVert}
%\providecommand{\norm}[1]{\lVert#1\rVert}
\providecommand{\mtx}[1]{\mathbf{#1}}
\providecommand{\mean}[1]{E\left[ #1 \right]}
\providecommand{\fourier}{\overset{\mathcal{F}}{ \rightleftharpoons}}
%\providecommand{\hilbert}{\overset{\mathcal{H}}{ \rightleftharpoons}}
\providecommand{\system}{\overset{\mathcal{H}}{ \longleftrightarrow}}
	%\newcommand{\solution}[2]{\textbf{Solution:}{#1}}
\newcommand{\solution}{\noindent \textbf{Solution: }}
\newcommand{\cosec}{\,\text{cosec}\,}
\providecommand{\dec}[2]{\ensuremath{\overset{#1}{\underset{#2}{\gtrless}}}}
\newcommand{\myvec}[1]{\ensuremath{\begin{pmatrix}#1\end{pmatrix}}}
\newcommand{\mydet}[1]{\ensuremath{\begin{vmatrix}#1\end{vmatrix}}}
%\numberwithin{equation}{section}
\numberwithin{equation}{subsection}
%\numberwithin{problem}{section}
%\numberwithin{definition}{section}
\makeatletter
\@addtoreset{figure}{problem}
\makeatother
\let\StandardTheFigure\thefigure
\let\vec\mathbf
%\renewcommand{\thefigure}{\theproblem.\arabic{figure}}
\renewcommand{\thefigure}{\theproblem}
%\setlist[enumerate,1]{before=\renewcommand\theequation{\theenumi.\arabic{equation}}
%\counterwithin{equation}{enumi}
%\renewcommand{\theequation}{\arabic{subsection}.\arabic{equation}}
\def\putbox#1#2#3{\makebox[0in][l]{\makebox[#1][l]{}\raisebox{\baselineskip}[0in][0in]{\raisebox{#2}[0in][0in]{#3}}}}
     \def\rightbox#1{\makebox[0in][r]{#1}}
     \def\centbox#1{\makebox[0in]{#1}}
     \def\topbox#1{\raisebox{-\baselineskip}[0in][0in]{#1}}
     \def\midbox#1{\raisebox{-0.5\baselineskip}[0in][0in]{#1}}
\vspace{3cm}
\title{Assignment 13}
\author{Venkatesh E\\AI20MTECH14005}
\maketitle
\newpage
%\tableofcontents
\bigskip
\renewcommand{\thefigure}{\theenumi}
\renewcommand{\thetable}{\theenumi}
\begin{abstract}
This document explains the concept of finding the solution to the system \vec{A}\vec{X}=\vec{Y} 
\end{abstract}
Download all latex-tikz codes from 
%
\begin{lstlisting}
https://github.com/venkateshelangovan/IIT-Hyderabad-Assignments/tree/master/Assignment13_Matrix_Theory
\end{lstlisting}
\section{Problem}
Let
\begin{align}
    \vec{A}=\myvec{3 & -6 & 2 & -1\\-2 & 4 & 1 & 3\\0 & 0 & 1& 1\\1 & -2 & 1 & 0} 
\end{align}
For which $(y_1,y_2,y_3,y_4)$ does the system of equations $\vec{A}\vec{X}=\vec{Y}$ have a solution ? 
\section{Solution}
Given , 
\begin{align}
    \vec{A}\vec{X}&=\vec{Y}\label{assume}\\
    \myvec{3 & -6 & 2 & -1\\-2 & 4 & 1 & 3\\0 & 0 & 1& 1\\1 & -2 & 1 & 0}\vec{X}&=\myvec{y_1 \\y_2\\y_3\\y_4}\label{given}
\end{align}
Now we try to find the matrix $\vec{B}$ such that $\vec{B}\vec{A}$ gives the row echelon form of matrix $\vec{A}$
Here,$\vec{B}$ is given by , 
\begin{align}
    \vec{B}&=\myvec{1&0&0&0\\\frac{2}{3}&1&0&0\\-\frac{2}{7}&-\frac{3}{7}&1&0\\0&\frac{1}{2}&-\frac{3}{2}&1}
\end{align}
\begin{align}
    \vec{B}\vec{A}&=\myvec{3 & -6 & 2 & -1\\0 & 0 & \frac{7}{3} & \frac{7}{3}\\0 & 0 & 0& 0\\0 & 0 & 0 & 0}\label{ref3}
\end{align}
Therefore, rank of matrix $\vec{A}$ is 2
Now $\vec{B}$ is expressed in terms of two block matrices
\begin{align}
    \vec{B}&=\myvec{\vec{B_1}\\\vec{B_2}}
\end{align}
\begin{align}
    \vec{B_1}&=\myvec{1&0&0&0\\\frac{2}{3}&1&0&0}\\
    \vec{B_2}&=\myvec{-\frac{2}{7}&-\frac{3}{7}&1&0\\0&\frac{1}{2}&-\frac{3}{2}&1}
\end{align}
Multiplying matrix $\vec{B}$ to both sides on the equation \eqref{assume}, we get , 
\begin{align}
    \myvec{\vec{B_1}\\\vec{B_2}}\vec{A}\vec{X}&=\myvec{\vec{B_1}\\\vec{B_2}}\vec{Y}\label{aug}
\end{align}
We know that , matrix $\vec{A}$ is of rank 2 
The augumented matrix of \eqref{aug} is given by 
\begin{align}
    \myvec{\vec{B_1}\vec{A} & \vrule & \vec{B_1}\vec{Y}\\
    \vec{B_2}\vec{A} & \vrule & \vec{B_2}\vec{Y}}
\end{align}
\begin{align}
    \vec{B_1}\vec{A}&=\myvec{3 & -6 & 2 & -1\\0 & 0 & \frac{7}{3} & \frac{7}{3}}\\
    \vec{B_2}\vec{A}&=\myvec{0 & 0 & 0 & 0\\ 0 & 0 & 0 & 0}
\end{align}
Since $\vec{B_2}\vec{A}$ is zero matrix and for the given system $\vec{A}\vec{X}=\vec{Y}$ to have a solution,
\begin{align}
\vec{B_2}\vec{Y}&=0\label{b1}\\
\myvec{-\frac{2}{7} & -\frac{3}{7} & 1 & 0 \\ 0 & \frac{1}{2} & -\frac{3}{2} & 1}\myvec{y_1\\y_2\\y_3\\y_4}&=0\label{b2y}
\end{align}
The augumented matrix of \eqref{b2y} is given by,
\begin{align}
    \myvec{-\frac{2}{7} & -\frac{3}{7} & 1 & 0 &\vrule & 0 \\ 0 & \frac{1}{2} & -\frac{3}{2} & 1 & \vrule & 0}
\end{align}
By row reduction technique,
\begin{align}
        &\xleftrightarrow{R_1=-\frac{7}{2}R_1}\myvec{1 & \frac{3}{2} &-\frac{7}{2}  & 0 &\vrule & 0 \\ 0 & \frac{1}{2} & -\frac{3}{2} & 1 & \vrule & 0}\\
        &\xleftrightarrow{R_2=2R_2}\myvec{1 & \frac{3}{2} &-\frac{7}{2}  & 0 &\vrule & 0 \\ 0 & 1 & -3 & 2 & \vrule & 0}\\
        &\xleftrightarrow{R_1=R_1-\frac{3}{2}R_2}\myvec{1 & 0 &1  & -3 &\vrule & 0 \\ 0 & 1 & -3 & 2 & \vrule & 0}
\end{align}
Equation \eqref{b2y} can be modified as , 
\begin{align}
    \myvec{1 & 0 & 1 & -3 \\ 0 & 1 & -3 & 2}\myvec{y_1\\y_2\\y_3\\y_4}&=0
\end{align}
Here $y_3$ and $y_4$ are free variables

If $y_3=a$ and $y_4=b$,then the solution to the system of equation $\vec{A}\vec{X}=\vec{Y}$ is given by,
\begin{align}
    \myvec{y_1\\y_2\\y_3\\y_4}&=a\myvec{-1\\3\\1\\0}+b\myvec{3\\-2\\0\\1}
\end{align}
One of the solution when $a=1$ and $b=2$  is given by , 
\begin{align}
    \myvec{y_1\\y_2\\y_3\\y_4}&=\myvec{-1\\3\\1\\0}+2\myvec{3\\-2\\0\\1}
\end{align}
\end{document}