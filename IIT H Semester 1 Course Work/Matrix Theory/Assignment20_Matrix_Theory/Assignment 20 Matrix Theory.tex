
\documentclass[journal,12pt]{IEEEtran}
\usepackage{longtable}
\usepackage{setspace}
\usepackage{gensymb}
\singlespacing
\usepackage[cmex10]{amsmath}
\newcommand\myemptypage{
\null
\thispagestyle{empty}
\addtocounter{page}{-1}
\newpage
}
\usepackage{amsthm}
\usepackage{mdframed}
\usepackage{mathrsfs}
\usepackage{txfonts}
\usepackage{stfloats}
\usepackage{bm}
\usepackage{cite}
\usepackage{cases}
\usepackage{subfig}

\usepackage{longtable}
\usepackage{multirow}

\usepackage{enumitem}
\usepackage{mathtools}
\usepackage{steinmetz}
\usepackage{tikz}
\usepackage{circuitikz}
\usepackage{verbatim}
\usepackage{tfrupee}
\usepackage[breaklinks=true]{hyperref}
\usepackage{graphicx}
\usepackage{tkz-euclide}

\usetikzlibrary{calc,math}
\usepackage{listings}
    \usepackage{color}                                            %%
    \usepackage{array}                                            %%
    \usepackage{longtable}                                        %%
    \usepackage{calc}                                             %%
    \usepackage{multirow}                                         %%
    \usepackage{hhline}                                           %%
    \usepackage{ifthen}                                           %%
    \usepackage{lscape}    
\usepackage{multicol}
\usepackage{chngcntr}

\DeclareMathOperator*{\Res}{Res}

\renewcommand\thesection{\arabic{section}}
\renewcommand\thesubsection{\thesection.\arabic{subsection}}
\renewcommand\thesubsubsection{\thesubsection.\arabic{subsubsection}}

\renewcommand\thesectiondis{\arabic{section}}
\renewcommand\thesubsectiondis{\thesectiondis.\arabic{subsection}}
\renewcommand\thesubsubsectiondis{\thesubsectiondis.\arabic{subsubsection}}


\hyphenation{op-tical net-works semi-conduc-tor}
\def\inputGnumericTable{}                                 %%

\lstset{
%language=C,
frame=single,
breaklines=true,
columns=fullflexible
}
\begin{document}
\onecolumn

\newtheorem{theorem}{Theorem}[section]
\newtheorem{problem}{Problem}
\newtheorem{proposition}{Proposition}[section]
\newtheorem{lemma}{Lemma}[section]
\newtheorem{corollary}[theorem]{Corollary}
\newtheorem{example}{Example}[section]
\newtheorem{definition}[problem]{Definition}

\newcommand{\BEQA}{\begin{eqnarray}}
\newcommand{\EEQA}{\end{eqnarray}}
\newcommand{\define}{\stackrel{\triangle}{=}}
\bibliographystyle{IEEEtran}
\raggedbottom
\setlength{\parindent}{0pt}
\providecommand{\mbf}{\mathbf}
\providecommand{\pr}[1]{\ensuremath{\Pr\left(#1\right)}}
\providecommand{\qfunc}[1]{\ensuremath{Q\left(#1\right)}}
\providecommand{\sbrak}[1]{\ensuremath{{}\left[#1\right]}}
\providecommand{\lsbrak}[1]{\ensuremath{{}\left[#1\right.}}
\providecommand{\rsbrak}[1]{\ensuremath{{}\left.#1\right]}}
\providecommand{\brak}[1]{\ensuremath{\left(#1\right)}}
\providecommand{\lbrak}[1]{\ensuremath{\left(#1\right.}}
\providecommand{\rbrak}[1]{\ensuremath{\left.#1\right)}}
\providecommand{\cbrak}[1]{\ensuremath{\left\{#1\right\}}}
\providecommand{\lcbrak}[1]{\ensuremath{\left\{#1\right.}}
\providecommand{\rcbrak}[1]{\ensuremath{\left.#1\right\}}}
\theoremstyle{remark}
\newtheorem{rem}{Remark}
\newcommand{\sgn}{\mathop{\mathrm{sgn}}}
%\providecommand{\hilbert}{\overset{\mathcal{H}}{ \rightleftharpoons}}
\providecommand{\system}{\overset{\mathcal{H}}{ \longleftrightarrow}}
%\newcommand{\solution}[2]{\textbf{Solution:}{#1}}
\newcommand{\solution}{\noindent \textbf{Solution: }}
\newcommand{\cosec}{\,\text{cosec}\,}
\providecommand{\dec}[2]{\ensuremath{\overset{#1}{\underset{#2}{\gtrless}}}}
\newcommand{\myvec}[1]{\ensuremath{\begin{pmatrix}#1\end{pmatrix}}}
\newcommand{\mydet}[1]{\ensuremath{\begin{vmatrix}#1\end{vmatrix}}}
\numberwithin{equation}{subsection}
\makeatletter
\@addtoreset{figure}{problem}
\makeatother
\let\StandardTheFigure\thefigure
\let\vec\mathbf
\renewcommand{\thefigure}{\theproblem}
\def\putbox#1#2#3{\makebox[0in][l]{\makebox[#1][l]{}\raisebox{\baselineskip}[0in][0in]{\raisebox{#2}[0in][0in]{#3}}}}
     \def\rightbox#1{\makebox[0in][r]{#1}}
     \def\centbox#1{\makebox[0in]{#1}}
     \def\topbox#1{\raisebox{-\baselineskip}[0in][0in]{#1}}
     \def\midbox#1{\raisebox{-0.5\baselineskip}[0in][0in]{#1}}
\vspace{3cm}
\title{Assignment 20}
\author{Venkatesh E\\AI20MTECH14005}
\maketitle
\bigskip
\renewcommand{\thefigure}{\theenumi}
\renewcommand{\thetable}{\theenumi}
Download latex-tikz codes from
\begin{lstlisting}
https://github.com/venkateshelangovan/IIT-Hyderabad-Assignments/tree/master/Assignment20_Matrix_Theory
\end{lstlisting}
\section{\textbf{Problem}}
Let $\vec{A}$ be an invertible $4 \times 4$ real matrix. Which of the following are NOT true ? 
\begin{enumerate}
\item  Rank $\vec{A}$ =4
\item For every vector $\vec{b} \in \mathbb{R}$, $\vec{A}\vec{x}=\vec{b}$ has exactly one solution. 
\item dim(nullspace $\vec{A}$)$\geq$ 1
\item 0 is an eigenvalue of $\vec{A}$
\end{enumerate}
\section{\textbf{Solution}}
\renewcommand{\thetable}{1}
\begin{longtable}{|l|l|}
\hline
\text{Given} & $\vec{A}$ is an invertible real matrix of order $4 \times 4$\\
\hline
\text{Solution} & \text{Since given $\vec{A}$ is an invertible matrix, \vec{A} has full rank.}\\
& \parbox{10cm}{\begin{align}
    det(\vec{A}) &\neq 0 \label{1}\\
    Rank(\vec{A})&=4 \label{s1}
\end{align}}\\
& \text{Let $\lambda_1,\lambda_2,\lambda_3$ and $\lambda_4$ be the eigenvalues of matrix $\vec{A}$}.\\
& We know that determinant of matrix $\vec{A}$ is the product of eigenvalues of $\vec{A}$.\\
& \parbox{10cm}{\begin{align}
    \lambda_1\lambda_2\lambda_3\lambda_4 \neq 0 \label{1}
\end{align}}\\
\hline
\textbf{Statement 1} & \parbox{10cm}{\begin{align}
    Rank(\vec{A})&=4\notag
\end{align}}\\
\hline
& \text{Since $\vec{A}$ is an invertible matrix,it has full rank as shown in equation \eqref{s1}.}\\
& \parbox{10cm}{\begin{center}
\textbf{True Statement }
\end{center}}\\
\hline
\textbf{Statement 2} & \text{For every vector $\vec{b} \in \mathbb{R}$, $\vec{A}\vec{x}=\vec{b}$ has exactly one solution.}\\
\hline
& \text{For every $\vec{b}$,}\\
& \parbox{10cm}{\begin{center}
    \vec{x}=\vec{A}$^{-1}$\vec{b}
  \end{center}}\\
& \text{$\vec{x}$ will be unique solution for every $\vec{b}$.}\\
& \parbox{10cm}{\begin{center}
\textbf{True Statement }
\end{center}}\\
\hline
\textbf{Statement 3} & \text{dim(nullspace $\vec{A}$)$\geq$ 1.}\\
\hline
& Using Rank Nullity Theorem,\\
& \parbox{10cm}{\begin{align}
    Rank(\vec{A})+dim(nullspace \vec{A})=n \notag\\
    \implies 4+dim(nullspace \vec{A})=4 \notag\\
    \implies dim(nullspace \vec{A})=0 \ngeq 1 \label{s3}
\end{align}}\\
\hline
& \text{where n is the number of columns in $\vec{A}$}\\
& \text{Equation \eqref{s3} proves that the given statement is \textbf{NOT True}.}\\
\hline
\textbf{Statement 4} & \text{0 is an eigenvalue of $\vec{A}$}\\
\hline
& From equation \eqref{1}, we could say that no eigenvalue of $\vec{A}$ could be 0.\\
& \parbox{10cm}{\begin{center}
\textbf{NOT True Statement }
\end{center}}\\
\hline
\caption{Explanation}
\label{table:1}
\end{longtable}
\end{document}