
\documentclass[journal,12pt]{IEEEtran}
\usepackage{longtable}
\usepackage{setspace}
\usepackage{gensymb}
\singlespacing
\usepackage[cmex10]{amsmath}
\newcommand\myemptypage{
\null
\thispagestyle{empty}
\addtocounter{page}{-1}
\newpage
}
\usepackage{amsthm}
\usepackage{mdframed}
\usepackage{mathrsfs}
\usepackage{txfonts}
\usepackage{stfloats}
\usepackage{bm}
\usepackage{cite}
\usepackage{cases}
\usepackage{subfig}

\usepackage{longtable}
\usepackage{multirow}

\usepackage{enumitem}
\usepackage{mathtools}
\usepackage{steinmetz}
\usepackage{tikz}
\usepackage{circuitikz}
\usepackage{verbatim}
\usepackage{tfrupee}
\usepackage[breaklinks=true]{hyperref}
\usepackage{graphicx}
\usepackage{tkz-euclide}

\usetikzlibrary{calc,math}
\usepackage{listings}
    \usepackage{color}                                            %%
    \usepackage{array}                                            %%
    \usepackage{longtable}                                        %%
    \usepackage{calc}                                             %%
    \usepackage{multirow}                                         %%
    \usepackage{hhline}                                           %%
    \usepackage{ifthen}                                           %%
    \usepackage{lscape}    
\usepackage{multicol}
\usepackage{chngcntr}

\DeclareMathOperator*{\Res}{Res}

\renewcommand\thesection{\arabic{section}}
\renewcommand\thesubsection{\thesection.\arabic{subsection}}
\renewcommand\thesubsubsection{\thesubsection.\arabic{subsubsection}}

\renewcommand\thesectiondis{\arabic{section}}
\renewcommand\thesubsectiondis{\thesectiondis.\arabic{subsection}}
\renewcommand\thesubsubsectiondis{\thesubsectiondis.\arabic{subsubsection}}


\hyphenation{op-tical net-works semi-conduc-tor}
\def\inputGnumericTable{}                                 %%

\lstset{
%language=C,
frame=single,
breaklines=true,
columns=fullflexible
}
\begin{document}
\onecolumn

\newtheorem{theorem}{Theorem}[section]
\newtheorem{problem}{Problem}
\newtheorem{proposition}{Proposition}[section]
\newtheorem{lemma}{Lemma}[section]
\newtheorem{corollary}[theorem]{Corollary}
\newtheorem{example}{Example}[section]
\newtheorem{definition}[problem]{Definition}

\newcommand{\BEQA}{\begin{eqnarray}}
\newcommand{\EEQA}{\end{eqnarray}}
\newcommand{\define}{\stackrel{\triangle}{=}}
\bibliographystyle{IEEEtran}
\raggedbottom
\setlength{\parindent}{0pt}
\providecommand{\mbf}{\mathbf}
\providecommand{\pr}[1]{\ensuremath{\Pr\left(#1\right)}}
\providecommand{\qfunc}[1]{\ensuremath{Q\left(#1\right)}}
\providecommand{\sbrak}[1]{\ensuremath{{}\left[#1\right]}}
\providecommand{\lsbrak}[1]{\ensuremath{{}\left[#1\right.}}
\providecommand{\rsbrak}[1]{\ensuremath{{}\left.#1\right]}}
\providecommand{\brak}[1]{\ensuremath{\left(#1\right)}}
\providecommand{\lbrak}[1]{\ensuremath{\left(#1\right.}}
\providecommand{\rbrak}[1]{\ensuremath{\left.#1\right)}}
\providecommand{\cbrak}[1]{\ensuremath{\left\{#1\right\}}}
\providecommand{\lcbrak}[1]{\ensuremath{\left\{#1\right.}}
\providecommand{\rcbrak}[1]{\ensuremath{\left.#1\right\}}}
\theoremstyle{remark}
\newtheorem{rem}{Remark}
\newcommand{\sgn}{\mathop{\mathrm{sgn}}}
%\providecommand{\hilbert}{\overset{\mathcal{H}}{ \rightleftharpoons}}
\providecommand{\system}{\overset{\mathcal{H}}{ \longleftrightarrow}}
%\newcommand{\solution}[2]{\textbf{Solution:}{#1}}
\newcommand{\solution}{\noindent \textbf{Solution: }}
\newcommand{\cosec}{\,\text{cosec}\,}
\providecommand{\dec}[2]{\ensuremath{\overset{#1}{\underset{#2}{\gtrless}}}}
\newcommand{\myvec}[1]{\ensuremath{\begin{pmatrix}#1\end{pmatrix}}}
\newcommand{\mydet}[1]{\ensuremath{\begin{vmatrix}#1\end{vmatrix}}}
\numberwithin{equation}{subsection}
\makeatletter
\@addtoreset{figure}{problem}
\makeatother
\let\StandardTheFigure\thefigure
\let\vec\mathbf
\renewcommand{\thefigure}{\theproblem}
\def\putbox#1#2#3{\makebox[0in][l]{\makebox[#1][l]{}\raisebox{\baselineskip}[0in][0in]{\raisebox{#2}[0in][0in]{#3}}}}
     \def\rightbox#1{\makebox[0in][r]{#1}}
     \def\centbox#1{\makebox[0in]{#1}}
     \def\topbox#1{\raisebox{-\baselineskip}[0in][0in]{#1}}
     \def\midbox#1{\raisebox{-0.5\baselineskip}[0in][0in]{#1}}
\vspace{3cm}
\title{Assignment 19}
\author{Venkatesh E\\AI20MTECH14005}
\maketitle
\bigskip
\renewcommand{\thefigure}{\theenumi}
\renewcommand{\thetable}{\theenumi}
Download latex-tikz codes from
\begin{lstlisting}
https://github.com/venkateshelangovan/IIT-Hyderabad-Assignments/tree/master/Assignment19_Matrix_Theory
\end{lstlisting}
\section{\textbf{Problem}}
Let $\vec{M}=\myvec{1 & -1 & 1\\2 & 1 & 4\\-2 & 1 & -4}$. Given that 1 is an eigenvalue of $\vec{M}$, then which of the following are correct ?
\begin{enumerate}
\item The minimal polynomial of $\vec{M}$ is $(x-1)(x+4)$  
\item The minimal polynomial of $\vec{M}$ is $(x-1)^2(x+4)$
\item $\vec{M}$ is not diagonalizable
\item $\vec{M}^{-1}=\frac{1}{4}(\vec{M}+3\vec{I})$
\end{enumerate}
\section{\textbf{Solution}}
\renewcommand{\thetable}{1}
\begin{longtable}{|l|l|}
\hline
\text{Given} & \parbox{10cm}{\begin{align}
    \vec{M}=\myvec{1 & -1 & 1\\2 & 1 & 4\\-2 & 1 & -4}
\end{align}}\\
& One of the eigenvalue of $\vec{M}$ is 1\\
\hline
\text{Solution} & \text{Let the eigenvalues of matrix $\vec{M}$ of order $3 \times 3$ be $\lambda_1,\lambda_2,\lambda_3$}\\
& \text{From given , let $\lambda_1=1$}.\\
& \text{We know that sum of the eigenvalues of matrix is Trace of the matrix and product of }\\
& \text{eigenvalues of matrix is Determinant of the matrix.}\\
& Trace of the square matrix(Tr($\vec{M}$)) is the sum of the elements in the main diagonal of $\vec{M}$.\\
& \parbox{10cm}{\begin{align}
    Tr(\vec{M})&=1+1-4\\
    \implies Tr(\vec{M})&=-2\\
    \implies \lambda_1+\lambda_2+\lambda_3&=-2\\
    \implies \lambda_2+\lambda_3&=-3\\
    \implies \lambda_2&=-3-\lambda_3 \label{1}
\end{align}}\\
& By row reducing the matrix $\vec{M}$, we get ,\\
& \parbox{10cm}{\begin{align}
 \vec{M}=\myvec{1 & -1 & 1 \\ 0 & 3 & 2\\ 0 & 0 & -\frac{4}{3}}
\end{align}}\\
\hline
& \parbox{10cm}{\begin{align}
    Det(\vec{M})&=1\left(3\left(-\frac{4}{3}\right)\right)=-4\\
    \implies \lambda_1\lambda_2\lambda_3&=-4\\
    \implies \lambda_2\lambda_3&=-4 \label{2}
\end{align}}\\
& \text{Solving equations \eqref{1} and \eqref{2} one of the possibilities we get, }\\
& \parbox{10cm}{\begin{align}
   \lambda_1&=1\\
   \lambda_2&=1\\
   \lambda_3&=-4
\end{align}}\\
\hline
& \text{Using the eigenvalues the characteristic polynomial of matrix $\vec{M}$ is given by,}\\
& \parbox{10cm}{\begin{align}
   c(x)&=x^3+2x^2-7x+4=0 \label{cx}
\end{align}}\\
& \text{The Cayley Hamilton Theorem states that every square matrix satisfies its own characteristic}\\
& \text{equation}.\\
& Using the above theorem, the equation \eqref{cx} can be written as,\\
& \parbox{10cm}{\begin{align}
   \vec{M}^3+2\vec{M}^2-7\vec{M}+4\vec{I}&=0 \label{cheq}\\
   \vec{M}^2+2\vec{M}-7\vec{I}+4\vec{M}^{-1}&=0\\
   \implies \vec{M}^{-1}&=-\frac{1}{4}(\vec{M}^2+2\vec{M}-7\vec{I}) \label{4}
\end{align}}\\
\hline
\textbf{Statement 1} & \text{The minimal polynomial of $\vec{M}$ is $(x-1)(x+4)$}\\
\hline
& \text{If (x-1)(x+4) is a minimal polynomial of \vec{M} then,}\\
& \parbox{10cm}{\begin{align}
   (\vec{M}-\vec{I})(\vec{M}+4\vec{I})&=\vec{0}_{3\times3}
\end{align}}\\
& \text{But,}\\
& \parbox{10cm}{\begin{align}
  (\vec{M}-\vec{I})(\vec{M}+4\vec{I})&=\myvec{-4 & -4 & -4\\2 & 2 & 2\\ 2 & 2 & 2}\neq \vec{0}_{3\times 3}
\end{align}}\\
& \parbox{10cm}{\begin{center}
\textbf{False Statement }
\end{center}}\\
\hline 
\textbf{Statement 2} & \text{The minimal polynomial of $\vec{M}$ is $(x-1)^2(x+4)$}\\
\hline
& Let m(x) be the minimal polynomial\\
& \parbox{10cm}{\begin{align}
   m(x)&=(x-1)^2(x+4) \label{mx}\\
   &=x^3+2x^2-7x+4\\
   &=c(x) \notag
\end{align}}\\
& \text{In this case both minimal polynomial and characteristic polynomial were same.} \\
& \text{Therefore wecould say that equation \eqref{mx} is the minimal polynomial of $\vec{M}$ as it satisfies }\\
& equation \eqref{cheq} by Cayley Hamilton Theorem. \\
& \parbox{10cm}{\begin{center}
\textbf{True Statement }
\end{center}}\\
\hline 
\textbf{Statement 3} & \text{$\vec{M}$ is not diagonalizable.} \\
\hline
& $\vec{M}$ is diagonalizable if and only if its minimal polynomial contains only linear factors.\\
\hline
& From equation \eqref{mx} we could see that one of the factor of minimal polynomial is \\
& repeated and it is not a linear factor. Therefore, Matrix $\vec{M}$ is not diagonalizable.\\
& \parbox{10cm}{\begin{center}
\textbf{True Statement }
\end{center}}\\
\hline 
\textbf{Statement 4} & \parbox{10cm}{\begin{align}
    \vec{M}^{-1}=\frac{1}{4}(\vec{M}+3\vec{I}) \label{minv}
\end{align}}\\
\hline
& \text{Comparing equation \eqref{4} and \eqref{minv} we could say that the given statement is}\\
& \text{\textbf{False Statement}}.\\
\hline
\caption{Explanation}
\label{table:1}
\end{longtable}
\end{document}