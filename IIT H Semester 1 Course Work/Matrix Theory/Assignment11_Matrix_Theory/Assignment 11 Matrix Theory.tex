\documentclass[journal,12pt,twocolumn]{IEEEtran}
%
\usepackage{setspace}
\usepackage{gensymb}
\usepackage{siunitx}
\usepackage{tkz-euclide} 
\usepackage{textcomp}
\usepackage{standalone}
\usetikzlibrary{calc}

%\doublespacing
\singlespacing

%\usepackage{graphicx}
%\usepackage{amssymb}
%\usepackage{relsize}
\usepackage[cmex10]{amsmath}
%\usepackage{amsthm}
%\interdisplaylinepenalty=2500
%\savesymbol{iint}
%\usepackage{txfonts}
%\restoresymbol{TXF}{iint}
%\usepackage{wasysym}
\usepackage{amsthm}
%\usepackage{iithtlc}
\usepackage{mathrsfs}
\usepackage{txfonts}
\usepackage{stfloats}
\usepackage{bm}
\usepackage{cite}
\usepackage{cases}
\usepackage{subfig}
%\usepackage{xtab}
\usepackage{longtable}
\usepackage{multirow}
%\usepackage{algorithm}
%\usepackage{algpseudocode}
\usepackage{enumitem}
\usepackage{mathtools}
\usepackage{steinmetz}
\usepackage{tikz}
\usepackage{circuitikz}
\usepackage{verbatim}
\usepackage{tfrupee}
\usepackage[breaklinks=true]{hyperref}
%\usepackage{stmaryrd}
\usepackage{tkz-euclide} % loads  TikZ and tkz-base
%\usetkzobj{all}
\usetikzlibrary{calc,math}
\usepackage{listings}
    \usepackage{color}                                            %%
    \usepackage{array}                                            %%
    \usepackage{longtable}                                        %%
    \usepackage{calc}                                             %%
    \usepackage{multirow}                                         %%
    \usepackage{hhline}                                           %%
    \usepackage{ifthen}                                           %%
  %optionally (for landscape tables embedded in another document): %%
    \usepackage{lscape}     
\usepackage{multicol}
\usepackage{chngcntr}
\usepackage{amsmath}
\usepackage{cleveref}
%\usepackage{enumerate}

%\usepackage{wasysym}
%\newcounter{MYtempeqncnt}
\DeclareMathOperator*{\Res}{Res}
%\renewcommand{\baselinestretch}{2}
\renewcommand\thesection{\arabic{section}}
\renewcommand\thesubsection{\thesection.\arabic{subsection}}
\renewcommand\thesubsubsection{\thesubsection.\arabic{subsubsection}}

\renewcommand\thesectiondis{\arabic{section}}
\renewcommand\thesubsectiondis{\thesectiondis.\arabic{subsection}}
\renewcommand\thesubsubsectiondis{\thesubsectiondis.\arabic{subsubsection}}

% correct bad hyphenation here
\hyphenation{op-tical net-works semi-conduc-tor}
\def\inputGnumericTable{}                                 %%

\lstset{
%language=C,
frame=single, 
breaklines=true,
columns=fullflexible
}
%\lstset{
%language=tex,
%frame=single, 
%breaklines=true
%}
\usepackage{graphicx}
\usepackage{pgfplots}

\begin{document}
%


\newtheorem{theorem}{Theorem}[section]
\newtheorem{problem}{Problem}
\newtheorem{proposition}{Proposition}[section]
\newtheorem{lemma}{Lemma}[section]
\newtheorem{corollary}[theorem]{Corollary}
\newtheorem{example}{Example}[section]
\newtheorem{definition}[problem]{Definition}
%\newtheorem{thm}{Theorem}[section] 
%\newtheorem{defn}[thm]{Definition}
%\newtheorem{algorithm}{Algorithm}[section]
%\newtheorem{cor}{Corollary}
\newcommand{\BEQA}{\begin{eqnarray}}
\newcommand{\EEQA}{\end{eqnarray}}
\newcommand{\define}{\stackrel{\triangle}{=}}
\bibliographystyle{IEEEtran}
%\bibliographystyle{ieeetr}
\providecommand{\mbf}{\mathbf}
\providecommand{\pr}[1]{\ensuremath{\Pr\left(#1\right)}}
\providecommand{\qfunc}[1]{\ensuremath{Q\left(#1\right)}}
\providecommand{\sbrak}[1]{\ensuremath{{}\left[#1\right]}}
\providecommand{\lsbrak}[1]{\ensuremath{{}\left[#1\right.}}
\providecommand{\rsbrak}[1]{\ensuremath{{}\left.#1\right]}}
\providecommand{\brak}[1]{\ensuremath{\left(#1\right)}}
\providecommand{\lbrak}[1]{\ensuremath{\left(#1\right.}}
\providecommand{\rbrak}[1]{\ensuremath{\left.#1\right)}}
\providecommand{\cbrak}[1]{\ensuremath{\left\{#1\right\}}}
\providecommand{\lcbrak}[1]{\ensuremath{\left\{#1\right.}}
\providecommand{\rcbrak}[1]{\ensuremath{\left.#1\right\}}}
\theoremstyle{remark}
\newtheorem{rem}{Remark}
\newcommand{\sgn}{\mathop{\mathrm{sgn}}}
\providecommand{\abs}[1]{\left\vert#1\right\vert}
\providecommand{\res}[1]{\Res\displaylimits_{#1}} 
\providecommand{\norm}[1]{\left\lVert#1\right\rVert}
%\providecommand{\norm}[1]{\lVert#1\rVert}
\providecommand{\mtx}[1]{\mathbf{#1}}
\providecommand{\mean}[1]{E\left[ #1 \right]}
\providecommand{\fourier}{\overset{\mathcal{F}}{ \rightleftharpoons}}
%\providecommand{\hilbert}{\overset{\mathcal{H}}{ \rightleftharpoons}}
\providecommand{\system}{\overset{\mathcal{H}}{ \longleftrightarrow}}
	%\newcommand{\solution}[2]{\textbf{Solution:}{#1}}
\newcommand{\solution}{\noindent \textbf{Solution: }}
\newcommand{\cosec}{\,\text{cosec}\,}
\providecommand{\dec}[2]{\ensuremath{\overset{#1}{\underset{#2}{\gtrless}}}}
\newcommand{\myvec}[1]{\ensuremath{\begin{pmatrix}#1\end{pmatrix}}}
\newcommand{\mydet}[1]{\ensuremath{\begin{vmatrix}#1\end{vmatrix}}}
%\numberwithin{equation}{section}
\numberwithin{equation}{subsection}
%\numberwithin{problem}{section}
%\numberwithin{definition}{section}
\makeatletter
\@addtoreset{figure}{problem}
\makeatother
\let\StandardTheFigure\thefigure
\let\vec\mathbf
%\renewcommand{\thefigure}{\theproblem.\arabic{figure}}
\renewcommand{\thefigure}{\theproblem}
%\setlist[enumerate,1]{before=\renewcommand\theequation{\theenumi.\arabic{equation}}
%\counterwithin{equation}{enumi}
%\renewcommand{\theequation}{\arabic{subsection}.\arabic{equation}}
\def\putbox#1#2#3{\makebox[0in][l]{\makebox[#1][l]{}\raisebox{\baselineskip}[0in][0in]{\raisebox{#2}[0in][0in]{#3}}}}
     \def\rightbox#1{\makebox[0in][r]{#1}}
     \def\centbox#1{\makebox[0in]{#1}}
     \def\topbox#1{\raisebox{-\baselineskip}[0in][0in]{#1}}
     \def\midbox#1{\raisebox{-0.5\baselineskip}[0in][0in]{#1}}
\vspace{3cm}
\title{Assignment 11}
\author{Venkatesh E\\AI20MTECH14005}
\maketitle
\newpage
%\tableofcontents
\bigskip
\renewcommand{\thefigure}{\theenumi}
\renewcommand{\thetable}{\theenumi}
\begin{abstract}
This document explains the proof that each subfield of the field of complex number contains every rational number
\end{abstract}
Download all latex-tikz codes from 
%
\begin{lstlisting}
https://github.com/venkateshelangovan/IIT-Hyderabad-Assignments/tree/master/Assignment11_Matrix_Theory
\end{lstlisting}
\section{Problem}
Prove that each subfield of the field of complex number contains every rational number
\section{Formal Defintion}
\subsection{Complex Numbers}
A complex number is a number that can be expressed in the form $a + bi$, where a and b are real numbers, and i represents the imaginary unit, satisfying the equation $i^2 =-1$.The set of complex numbers is denoted by $\mathbb{C}$
\begin{align}
    \mathbb{C}=\{(a,b):a,b \in \mathbb{R}\}
\end{align}
\subsection{Rational Numbers}
A number in the form $\frac{p}{q}$, where both p and q(non-zero) are integers, is called a rational number.The set of rational numbers is dentoed by $\mathbb{Q}$
\section{Proof}
Let $\mathbb{Q}$ be the set of rational numbers.
\begin{align}
    \mathbb{Q}=\left\{\frac{p}{q}:p \in \mathbb{Z},q \in \mathbb{Z}_{\not=0} \right\}\label{q}
\end{align}
Let $\mathbb{C}$ be the field of complex numbers and given $\mathbb{F}$ be the subfield of field of complex numbers $\mathbb{C}$ 
Since $\mathbb{F}$ is the subfield , we could say that 
\begin{align}
    0 &\in \mathbb{F} \label{0}\\
    1 &\in \mathbb{F}
\end{align}
\subsection{Closed under addition}
Here $\mathbb{F}$ is closed under addition since it is subfield
\begin{align}
    1+1=2&\in \mathbb{F}\\
    1+1+1=3&\in \mathbb{F}\\
    \vdots\notag\\
    1+1+\dots+1\text{(p times)}= p &\in \mathbb{F}\label{p}\\
    1+1+\dots+1\text{(q times)}= q &\in \mathbb{F}\label{q1}
\end{align}
By using the above property we could say that zero and other positive integers belongs to $\mathbb{F}$.Since $p$ and $q$ are integers we say,
\begin{align}
    p \in \mathbb{Z}\\
    q \in \mathbb{Z}\label{0}
\end{align}
\subsection{Additive Inverse}
Let $x$ be the positive integer belong $\mathbb{F}$ and by additive inverse we could say, 
\begin{align}
    \forall x &\in \mathbb{F}\label{1}\\
    (-x) &\in \mathbb{F} \label{2}
\end{align}
Therefore field $\mathbb{F}$ contains every integers. Let n be a integer then,
\begin{align}
    n \in \mathbb{Z} &\implies n \in \mathbb{F}\\
    \mathbb{Z} &\subseteq \mathbb{F}
\end{align}
Where $\mathbb{Z}$ is subset of $\mathbb{F}$
\subsection{Multiplicative Inverse}
Every element except zero in the subfield $\mathbb{F}$ has an multiplicative inverse. From equation \eqref{q1}, since q $&\in \mathbb{F}$ we could say ,
\begin{align}
    \frac{1}{q} \in \mathbb{F} \quad{\text{and  }} q \not= 0\label{3}
\end{align}
\subsection{Closed under multiplication}
Also, $\mathbb{F}$ is closed under multiplication and thus,
from equation \eqref{p} and \eqref{3} we get , 
\begin{align}
    p\cdot\frac{1}{q} \in \mathbb{F}\\
\implies \frac{p}{q} \in \mathbb{F}\label{proof}
\end{align}
where , $p \in \mathbb{Z}$ and $q \in \mathbb{Z}_{\not=0}$ (from equation \eqref{0} and \eqref{3})
\subsection{Conclusion}
From \eqref{q} and \eqref{proof} we could say , 
\begin{align}
    \mathbb{Q} \subseteq \mathbb{F}\label{F}
\end{align}
From equation \eqref{F} we could say that each subfield of the field of complex number contains every rational number
\begin{center}
    Hence Proved
\end{center}
\end{document}