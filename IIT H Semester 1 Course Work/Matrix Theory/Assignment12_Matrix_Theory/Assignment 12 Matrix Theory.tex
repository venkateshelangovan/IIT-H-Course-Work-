\documentclass[journal,12pt,twocolumn]{IEEEtran}
%
\usepackage{setspace}
\usepackage{gensymb}
\usepackage{siunitx}
\usepackage{tkz-euclide} 
\usepackage{textcomp}
\usepackage{standalone}
\usetikzlibrary{calc}

%\doublespacing
\singlespacing

%\usepackage{graphicx}
%\usepackage{amssymb}
%\usepackage{relsize}
\usepackage[cmex10]{amsmath}
%\usepackage{amsthm}
%\interdisplaylinepenalty=2500
%\savesymbol{iint}
%\usepackage{txfonts}
%\restoresymbol{TXF}{iint}
%\usepackage{wasysym}
\usepackage{amsthm}
%\usepackage{iithtlc}
\usepackage{mathrsfs}
\usepackage{txfonts}
\usepackage{stfloats}
\usepackage{bm}
\usepackage{cite}
\usepackage{cases}
\usepackage{subfig}
%\usepackage{xtab}
\usepackage{longtable}
\usepackage{multirow}
%\usepackage{algorithm}
%\usepackage{algpseudocode}
\usepackage{enumitem}
\usepackage{mathtools}
\usepackage{steinmetz}
\usepackage{tikz}
\usepackage{circuitikz}
\usepackage{verbatim}
\usepackage{tfrupee}
\usepackage[breaklinks=true]{hyperref}
%\usepackage{stmaryrd}
\usepackage{tkz-euclide} % loads  TikZ and tkz-base
%\usetkzobj{all}
\usetikzlibrary{calc,math}
\usepackage{listings}
    \usepackage{color}                                            %%
    \usepackage{array}                                            %%
    \usepackage{longtable}                                        %%
    \usepackage{calc}                                             %%
    \usepackage{multirow}                                         %%
    \usepackage{hhline}                                           %%
    \usepackage{ifthen}                                           %%
  %optionally (for landscape tables embedded in another document): %%
    \usepackage{lscape}     
\usepackage{multicol}
\usepackage{chngcntr}
\usepackage{amsmath}
\usepackage{cleveref}
%\usepackage{enumerate}

%\usepackage{wasysym}
%\newcounter{MYtempeqncnt}
\DeclareMathOperator*{\Res}{Res}
%\renewcommand{\baselinestretch}{2}
\renewcommand\thesection{\arabic{section}}
\renewcommand\thesubsection{\thesection.\arabic{subsection}}
\renewcommand\thesubsubsection{\thesubsection.\arabic{subsubsection}}

\renewcommand\thesectiondis{\arabic{section}}
\renewcommand\thesubsectiondis{\thesectiondis.\arabic{subsection}}
\renewcommand\thesubsubsectiondis{\thesubsectiondis.\arabic{subsubsection}}

% correct bad hyphenation here
\hyphenation{op-tical net-works semi-conduc-tor}
\def\inputGnumericTable{}                                 %%

\lstset{
%language=C,
frame=single, 
breaklines=true,
columns=fullflexible
}
%\lstset{
%language=tex,
%frame=single, 
%breaklines=true
%}
\usepackage{graphicx}
\usepackage{pgfplots}

\begin{document}
%


\newtheorem{theorem}{Theorem}[section]
\newtheorem{problem}{Problem}
\newtheorem{proposition}{Proposition}[section]
\newtheorem{lemma}{Lemma}[section]
\newtheorem{corollary}[theorem]{Corollary}
\newtheorem{example}{Example}[section]
\newtheorem{definition}[problem]{Definition}
%\newtheorem{thm}{Theorem}[section] 
%\newtheorem{defn}[thm]{Definition}
%\newtheorem{algorithm}{Algorithm}[section]
%\newtheorem{cor}{Corollary}
\newcommand{\BEQA}{\begin{eqnarray}}
\newcommand{\EEQA}{\end{eqnarray}}
\newcommand{\define}{\stackrel{\triangle}{=}}
\bibliographystyle{IEEEtran}
%\bibliographystyle{ieeetr}
\providecommand{\mbf}{\mathbf}
\providecommand{\pr}[1]{\ensuremath{\Pr\left(#1\right)}}
\providecommand{\qfunc}[1]{\ensuremath{Q\left(#1\right)}}
\providecommand{\sbrak}[1]{\ensuremath{{}\left[#1\right]}}
\providecommand{\lsbrak}[1]{\ensuremath{{}\left[#1\right.}}
\providecommand{\rsbrak}[1]{\ensuremath{{}\left.#1\right]}}
\providecommand{\brak}[1]{\ensuremath{\left(#1\right)}}
\providecommand{\lbrak}[1]{\ensuremath{\left(#1\right.}}
\providecommand{\rbrak}[1]{\ensuremath{\left.#1\right)}}
\providecommand{\cbrak}[1]{\ensuremath{\left\{#1\right\}}}
\providecommand{\lcbrak}[1]{\ensuremath{\left\{#1\right.}}
\providecommand{\rcbrak}[1]{\ensuremath{\left.#1\right\}}}
\theoremstyle{remark}
\newtheorem{rem}{Remark}
\newcommand{\sgn}{\mathop{\mathrm{sgn}}}
\providecommand{\abs}[1]{\left\vert#1\right\vert}
\providecommand{\res}[1]{\Res\displaylimits_{#1}} 
\providecommand{\norm}[1]{\left\lVert#1\right\rVert}
%\providecommand{\norm}[1]{\lVert#1\rVert}
\providecommand{\mtx}[1]{\mathbf{#1}}
\providecommand{\mean}[1]{E\left[ #1 \right]}
\providecommand{\fourier}{\overset{\mathcal{F}}{ \rightleftharpoons}}
%\providecommand{\hilbert}{\overset{\mathcal{H}}{ \rightleftharpoons}}
\providecommand{\system}{\overset{\mathcal{H}}{ \longleftrightarrow}}
	%\newcommand{\solution}[2]{\textbf{Solution:}{#1}}
\newcommand{\solution}{\noindent \textbf{Solution: }}
\newcommand{\cosec}{\,\text{cosec}\,}
\providecommand{\dec}[2]{\ensuremath{\overset{#1}{\underset{#2}{\gtrless}}}}
\newcommand{\myvec}[1]{\ensuremath{\begin{pmatrix}#1\end{pmatrix}}}
\newcommand{\mydet}[1]{\ensuremath{\begin{vmatrix}#1\end{vmatrix}}}
%\numberwithin{equation}{section}
\numberwithin{equation}{subsection}
%\numberwithin{problem}{section}
%\numberwithin{definition}{section}
\makeatletter
\@addtoreset{figure}{problem}
\makeatother
\let\StandardTheFigure\thefigure
\let\vec\mathbf
%\renewcommand{\thefigure}{\theproblem.\arabic{figure}}
\renewcommand{\thefigure}{\theproblem}
%\setlist[enumerate,1]{before=\renewcommand\theequation{\theenumi.\arabic{equation}}
%\counterwithin{equation}{enumi}
%\renewcommand{\theequation}{\arabic{subsection}.\arabic{equation}}
\def\putbox#1#2#3{\makebox[0in][l]{\makebox[#1][l]{}\raisebox{\baselineskip}[0in][0in]{\raisebox{#2}[0in][0in]{#3}}}}
     \def\rightbox#1{\makebox[0in][r]{#1}}
     \def\centbox#1{\makebox[0in]{#1}}
     \def\topbox#1{\raisebox{-\baselineskip}[0in][0in]{#1}}
     \def\midbox#1{\raisebox{-0.5\baselineskip}[0in][0in]{#1}}
\vspace{3cm}
\title{Assignment 12}
\author{Venkatesh E\\AI20MTECH14005}
\maketitle
\newpage
%\tableofcontents
\bigskip
\renewcommand{\thefigure}{\theenumi}
\renewcommand{\thetable}{\theenumi}
\begin{abstract}
This document explains the proof of complex entries in $2\times2$ matrices 
\end{abstract}
Download all latex-tikz codes from 
%
\begin{lstlisting}
https://github.com/venkateshelangovan/IIT-Hyderabad-Assignments/tree/master/Assignment12_Matrix_Theory
\end{lstlisting}
\section{Problem}
Let
\begin{align}
    \vec{A}=\myvec{a & b\\c & d}
\end{align}
be a $2\times2$ matrix with complex entries. Suppose A is row-reduced and also that $a+b+c+d=0$. Prove that there are exactly three such matrices. 
\section{Definition}
\subsection{Row Echelon Form}
A matrix is in row echelon  form if it follows the following conditions

1. All nonzero rows are above any rows of all zeros.

2. Each leading entry (i.e. left most nonzero entry) of a row is in a column to the right of the leading entry of the row above it.

3. All entries in a column below a leading entry are zero
\subsection{Row Reduced Echelon Form}
A matrix is in row reduced echelon form if it follows the following conditions

1. The matrix should be row echelon form 

2. The leading entry in each nonzero row is 1.

3. Each leading 1 is the only nonzero entry in its column.
\section{Proof}

Given ,
\begin{align}
    \vec{A}=\myvec{a & b \\c & d}\label{given}
\end{align}

\textbf{Condition 1 :} Matrix $\vec{A}$ should be in row-reduced echelon form 

\textbf{Condition 2 :}  $a+b+c+d=0$ where a,b,c and d are the elements of the matrix $\vec{A}$

Reducing the matrix $\vec{A}$ from equation \eqref{given}
\begin{align}
\myvec{a&b\\c&d}&\xleftrightarrow{R_1=\frac{1}{a}R_1}\myvec{1&\frac{b}{a}\\c&d}\\
&\xleftrightarrow{R_2 = R_2-cR_1}\myvec{1&\frac{b}{a}\\0&\frac{ad-bc}{a}}\label{x}\\
&\xleftrightarrow{R_2 = \frac{a}{ad-bc}R_2}\myvec{1&\frac{b}{a}\\0&1}\\
&\xleftrightarrow{R_1 = R_1-\frac{b}{a}R_2}\myvec{1&0\\0&1}\label{change}
\end{align}

\textbf{Case 1}: Matrix $\vec{A}$ of Rank 2

From the equation \eqref{x},for the matrix to be in row reduced echelon form,
\begin{align}
    b&=0\notag\\
    a&\not=0\notag\\
    d&=1\notag\\
    \vec{A}&=\myvec{1 & 0 \\ 0 & 1}\label{r2}
\end{align}
For the condition 2 to get satisfied,
\begin{align}
    a+0+c+1&=0\\
    \implies a&=-(c+1)\\
    \implies c&\not=-1
\end{align}

Both the condition gets satisfied and so exactly one matrix $\vec{A}$ can be formed of Rank 2 with given conditions 

\textbf{Case 2}: Matrix $\vec{A}$ of Rank 1

From the equation \eqref{x},for the matrix to be in row reduced echelon form,
\begin{align}
    a&\not=0\notag\\
    d&=0\notag\\
    c&=0\notag
\end{align}
For the condition 2 to get satisfied, 
\begin{align}
    a+b+0+0&=0\\
    \implies b&=-a
\end{align}
\begin{align}
    \vec{A}=\myvec{1 & -1\\ 0 & 0}\label{r1}
\end{align}

Both the condition gets satisfied and so exactly one matrix $\vec{A}$ can be formed of Rank 1 with given conditions 

\textbf{Case 3}: Matrix $\vec{A}$ of Rank 0

From equation \eqref{given},for the matrix to be in row reduced echelon form,
\begin{align}
    a&=0\notag\\
    b&=0\notag\\
    c&=0\notag\\
    d&=0\notag\\
    \vec{A}&=\myvec{0 & 0\\ 0 & 0}\label{r0}
\end{align}
Both the condition gets satisfied and so exactly one matrix $\vec{A}$ can be formed of Rank 0 with given conditions 

Therefore matrix $\vec{A}$ shown in equation \eqref{r2},\eqref{r1} and \eqref{r0} are the exactly three such matrices that can be formed with given conditions.
\begin{center}
    Hence Proved
\end{center}
\end{document}