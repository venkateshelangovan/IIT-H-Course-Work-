\documentclass[journal,12pt,twocolumn]{IEEEtran}
%
\usepackage{setspace}
\usepackage{gensymb}
\usepackage{siunitx}
\usepackage{tkz-euclide} 
\usepackage{textcomp}
\usepackage{standalone}
\usetikzlibrary{calc}

%\doublespacing
\singlespacing

%\usepackage{graphicx}
%\usepackage{amssymb}
%\usepackage{relsize}
\usepackage[cmex10]{amsmath}
%\usepackage{amsthm}
%\interdisplaylinepenalty=2500
%\savesymbol{iint}
%\usepackage{txfonts}
%\restoresymbol{TXF}{iint}
%\usepackage{wasysym}
\usepackage{amsthm}
%\usepackage{iithtlc}
\usepackage{mathrsfs}
\usepackage{txfonts}
\usepackage{stfloats}
\usepackage{bm}
\usepackage{cite}
\usepackage{cases}
\usepackage{subfig}
%\usepackage{xtab}
\usepackage{longtable}
\usepackage{multirow}
%\usepackage{algorithm}
%\usepackage{algpseudocode}
\usepackage{enumitem}
\usepackage{mathtools}
\usepackage{steinmetz}
\usepackage{tikz}
\usepackage{circuitikz}
\usepackage{verbatim}
\usepackage{tfrupee}
\usepackage[breaklinks=true]{hyperref}
%\usepackage{stmaryrd}
\usepackage{tkz-euclide} % loads  TikZ and tkz-base
%\usetkzobj{all}
\usetikzlibrary{calc,math}
\usepackage{listings}
    \usepackage{color}                                            %%
    \usepackage{array}                                            %%
    \usepackage{longtable}                                        %%
    \usepackage{calc}                                             %%
    \usepackage{multirow}                                         %%
    \usepackage{hhline}                                           %%
    \usepackage{ifthen}                                           %%
  %optionally (for landscape tables embedded in another document): %%
    \usepackage{lscape}     
\usepackage{multicol}
\usepackage{chngcntr}
\usepackage{amsmath}
\usepackage{cleveref}
%\usepackage{enumerate}

%\usepackage{wasysym}
%\newcounter{MYtempeqncnt}
\DeclareMathOperator*{\Res}{Res}
%\renewcommand{\baselinestretch}{2}
\renewcommand\thesection{\arabic{section}}
\renewcommand\thesubsection{\thesection.\arabic{subsection}}
\renewcommand\thesubsubsection{\thesubsection.\arabic{subsubsection}}

\renewcommand\thesectiondis{\arabic{section}}
\renewcommand\thesubsectiondis{\thesectiondis.\arabic{subsection}}
\renewcommand\thesubsubsectiondis{\thesubsectiondis.\arabic{subsubsection}}

% correct bad hyphenation here
\hyphenation{op-tical net-works semi-conduc-tor}
\def\inputGnumericTable{}                                 %%

\lstset{
%language=C,
frame=single, 
breaklines=true,
columns=fullflexible
}
%\lstset{
%language=tex,
%frame=single, 
%breaklines=true
%}
\usepackage{graphicx}
\usepackage{pgfplots}

\begin{document}
%


\newtheorem{theorem}{Theorem}[section]
\newtheorem{problem}{Problem}
\newtheorem{proposition}{Proposition}[section]
\newtheorem{lemma}{Lemma}[section]
\newtheorem{corollary}[theorem]{Corollary}
\newtheorem{example}{Example}[section]
\newtheorem{definition}[problem]{Definition}
%\newtheorem{thm}{Theorem}[section] 
%\newtheorem{defn}[thm]{Definition}
%\newtheorem{algorithm}{Algorithm}[section]
%\newtheorem{cor}{Corollary}
\newcommand{\BEQA}{\begin{eqnarray}}
\newcommand{\EEQA}{\end{eqnarray}}
\newcommand{\define}{\stackrel{\triangle}{=}}
\bibliographystyle{IEEEtran}
%\bibliographystyle{ieeetr}
\providecommand{\mbf}{\mathbf}
\providecommand{\pr}[1]{\ensuremath{\Pr\left(#1\right)}}
\providecommand{\qfunc}[1]{\ensuremath{Q\left(#1\right)}}
\providecommand{\sbrak}[1]{\ensuremath{{}\left[#1\right]}}
\providecommand{\lsbrak}[1]{\ensuremath{{}\left[#1\right.}}
\providecommand{\rsbrak}[1]{\ensuremath{{}\left.#1\right]}}
\providecommand{\brak}[1]{\ensuremath{\left(#1\right)}}
\providecommand{\lbrak}[1]{\ensuremath{\left(#1\right.}}
\providecommand{\rbrak}[1]{\ensuremath{\left.#1\right)}}
\providecommand{\cbrak}[1]{\ensuremath{\left\{#1\right\}}}
\providecommand{\lcbrak}[1]{\ensuremath{\left\{#1\right.}}
\providecommand{\rcbrak}[1]{\ensuremath{\left.#1\right\}}}
\theoremstyle{remark}
\newtheorem{rem}{Remark}
\newcommand{\sgn}{\mathop{\mathrm{sgn}}}
\providecommand{\abs}[1]{\left\vert#1\right\vert}
\providecommand{\res}[1]{\Res\displaylimits_{#1}} 
\providecommand{\norm}[1]{\left\lVert#1\right\rVert}
%\providecommand{\norm}[1]{\lVert#1\rVert}
\providecommand{\mtx}[1]{\mathbf{#1}}
\providecommand{\mean}[1]{E\left[ #1 \right]}
\providecommand{\fourier}{\overset{\mathcal{F}}{ \rightleftharpoons}}
%\providecommand{\hilbert}{\overset{\mathcal{H}}{ \rightleftharpoons}}
\providecommand{\system}{\overset{\mathcal{H}}{ \longleftrightarrow}}
	%\newcommand{\solution}[2]{\textbf{Solution:}{#1}}
\newcommand{\solution}{\noindent \textbf{Solution: }}
\newcommand{\cosec}{\,\text{cosec}\,}
\providecommand{\dec}[2]{\ensuremath{\overset{#1}{\underset{#2}{\gtrless}}}}
\newcommand{\myvec}[1]{\ensuremath{\begin{pmatrix}#1\end{pmatrix}}}
\newcommand{\mydet}[1]{\ensuremath{\begin{vmatrix}#1\end{vmatrix}}}
%\numberwithin{equation}{section}
\numberwithin{equation}{subsection}
%\numberwithin{problem}{section}
%\numberwithin{definition}{section}
\makeatletter
\@addtoreset{figure}{problem}
\makeatother
\let\StandardTheFigure\thefigure
\let\vec\mathbf
%\renewcommand{\thefigure}{\theproblem.\arabic{figure}}
\renewcommand{\thefigure}{\theproblem}
%\setlist[enumerate,1]{before=\renewcommand\theequation{\theenumi.\arabic{equation}}
%\counterwithin{equation}{enumi}
%\renewcommand{\theequation}{\arabic{subsection}.\arabic{equation}}
\def\putbox#1#2#3{\makebox[0in][l]{\makebox[#1][l]{}\raisebox{\baselineskip}[0in][0in]{\raisebox{#2}[0in][0in]{#3}}}}
     \def\rightbox#1{\makebox[0in][r]{#1}}
     \def\centbox#1{\makebox[0in]{#1}}
     \def\topbox#1{\raisebox{-\baselineskip}[0in][0in]{#1}}
     \def\midbox#1{\raisebox{-0.5\baselineskip}[0in][0in]{#1}}
\vspace{3cm}
\title{Assignment 14}
\author{Venkatesh E\\AI20MTECH14005}
\maketitle
\newpage
%\tableofcontents
\bigskip
\renewcommand{\thefigure}{\theenumi}
\renewcommand{\thetable}{\theenumi}
\begin{abstract}
This document explains the proof such that if $\vec{A}$ is an $m\times n$ matrix and $\vec{B}$ is an $n\times m$ matrix and $n<m$,then $\vec{A}\vec{B}$ is not invertible
\end{abstract}
Download all latex-tikz codes from 
%
\begin{lstlisting}
https://github.com/venkateshelangovan/IIT-Hyderabad-Assignments/tree/master/Assignment14_Matrix_Theory
\end{lstlisting}
\section{Problem}
Prove that if  $\vec{A}$ is an $m\times n$ matrix, $\vec{B}$ is an $n\times m$ matrix and $n<m$,then $\vec{A}\vec{B}$ is not invertible
\section{Rank of a Matrix}
\subsection{Definition}
The rank of a matrix is defined as 

1. The maximum number of linearly independent column vectors in the matrix or

2. The maximum number of linearly independent row vectors in the matrix.
\subsection{To prove Row Rank=Column Rank}
Consider the matrix $\vec{A}$ of order $m\times n$,  

The row rank is the maximum number of linearly independent rows in the matrix $\vec{A}$
\begin{align}
    RowRank(\vec{A}) \leq m \label{rr}
\end{align}
The column rank is the maximum number of linearly independent column in the matrix $\vec{A}$
\begin{align}
    ColumnRank(\vec{A}) \leq n \label{cr}
\end{align}
A matrix $\vec{P_n}$ is a permutation matrix of order $n \times n$ if and only if it is obtained from $n \times n$ Identity matrix $\vec{I_n}$ by performing one or more interchanges of the rows and columns of $\vec{I_n}$.

One of the $3 \times 3$ permutation matrix is given by , 
\begin{align}
    \vec{P_3}=\myvec{1 & 0 & 0 \\0 & 0 & 1\\ 0 & 1 & 0}
\end{align}
Let $\vec{P_n}$ be an $n \times n$ permuation matrix,
\begin{align}
    \vec{A}\vec{P_n}=\myvec{\vec{X}&\vec{W}}\label{p1}
\end{align}
where  columns of $\vec{X}$ are the d pivot columns of $\vec{A}$.

Every column of $\vec{W}$ is a linear combination of the columns of $\vec{X}$, so there is a matrix $\vec{K}$ such that,
\begin{align}
    \vec{W}=\vec{X}\vec{K}\label{p2}
\end{align}
where the columns of $\vec{K}$ contain the coefficients of each of those linear combinations.

Substituting the equation \eqref{p2} in equation \eqref{p1},we get 
\begin{align}
    \vec{A}\vec{P_n}&=\myvec{\vec{X} &\vec{X}\vec{K}}\\
    \implies \vec{A}\vec{P_n}&=\vec{X}\myvec{\vec{I_d}&\vec{K}}\label{p3}
\end{align}
where $\vec{I_d}$ represents the $d \times d$ identity matrix

Transforming the matrix $\vec{A}$ into reduced row echelon form, we get ,
\begin{align}
    \vec{B}=\vec{E}\vec{A}
\end{align}
where $\vec{E}$ is the product of elementary matrices and $\vec{B}$ is the reduced row echelon form of $\vec{A}$

Multiplying $\vec{E}$ to the equation \eqref{p3} on both sides,
\begin{align}
    \vec{E}\vec{A}\vec{P_n}&=\vec{E}\vec{X}\myvec{\vec{I_d}&\vec{K}}\\
    \implies \vec{B}\vec{P_n}&=\vec{E}\vec{X}\myvec{\vec{I_d}&\vec{K}}\label{p5}
\end{align}
Where, 
\begin{align}
    \vec{E}\vec{X}=\myvec{\vec{I_d}\\0}\label{p4}
\end{align}
Substituting the equation \eqref{p4} in equation \eqref{p5},
\begin{align}
    \vec{B}\vec{P_n}=\myvec{\vec{I_d} & \vec{K} \\ 0 & 0}
\end{align}
Here ,we could see that the nonzero d rows of the reduced row echelon form with the same permutation on the columns as we did for $\vec{A}$.Therefore we could say that, 
\begin{align}
    \myvec{\vec{I_d} & \vec{K}}=\vec{Y}\vec{P_n}\label{p6}
\end{align}
Substituting the equation \eqref{p6} in equation \eqref{p3},we get ,
\begin{align}
    \vec{A}\vec{P_n}=\vec{X}\vec{Y}\vec{P_n}
\end{align}
Here $\vec{P_n}$ is a permutation matrix and it is invertible.This implies,
\begin{align}
    \vec{A}=\vec{X}\vec{Y}
\end{align}
where matrix $\vec{X}$ is of order $m\times d$ and matrix $\vec{Y}$ is of order $d\times n$

Consider the example , 
\begin{align}
    \vec{A}&=\myvec{4 & 8 & 9 \\ 2 & 4 & 7}\label{af}\\
    \vec{A}\vec{P_3}&=\myvec{4 & 9 & 8\\2 & 7 & 4}
\end{align}
Let,
\begin{align}
    \vec{X}&=\myvec{4 & 9 \\ 2 & 7}\label{xf}\\
    \vec{W}&=\vec{X}\vec{K}=\myvec{8 \\ 4}\label{w}\\ 
    \vec{I_2}&=\myvec{1 & 0 \\ 0 & 1}
\end{align}
\begin{align}
    \vec{A}\vec{P_3}&=\myvec{\vec{X} & \vec{W}}\\
    \vec{A}\vec{P_3}&=\myvec{\vec{X} & \vec{X}\vec{K}}\\
    \vec{A}\vec{P_3}&=\vec{X}\myvec{\vec{I_2} & \vec{K}}
\end{align}
From equation \eqref{w},
\begin{align}
    \vec{K}&=\vec{X}^{-1}\vec{W}\\
    \vec{K}&=\myvec{2 \\0}
\end{align}
Let $\vec{B}$ be reduced row echelon form of matrix $\vec{A}$,
\begin{align}
\vec{B}&=\myvec{1 & 2 & 0 \\0 & 0 & 1}\\
\vec{B}\vec{P_n}&=\myvec{1 & 0 & 2 \\ 0 & 1 & 0}
\end{align}
Let,
\begin{align}
    \myvec{\vec{I_2} & \vec{K}}&=\vec{Y}\vec{P_n}\\
    \myvec{1 & 0 & 2\\0 & 1 & 0}&=\vec{Y}\myvec{1 & 0 & 0\\0 & 0 &1\\0 & 1 & 0}\\
    \implies \vec{Y}&=\myvec{1 & 2 & 0 \\0 & 0 & 1}\label{yf}
\end{align}
From the above equations \eqref{af},\eqref{xf} and \eqref{yf}, it can be seen that ,
\begin{align}
    \vec{A}=\vec{X}\vec{Y}\notag
\end{align}
Now for the matrix $\vec{A}$ of order $m \times n$,

Every row of $\vec{A}$ is the linear combination of the rows of $\vec{Y}$
\begin{align}
    rowspace(\vec{A})\subsetneq span(\{\vec{Y_1},\vec{Y_2},\dots,\vec{Y_d}\})
\end{align}
where $\vec{Y_1},\vec{Y_2},\dots,\vec{Y_d}$ were the rows of $\vec{Y}$

The dimension of the row space is atmost d

Every column of $\vec{A}$ is the linear combination of the columns of $\vec{X}$
\begin{align}
    columnspace(\vec{A})\subsetneq span(\{\vec{X_1},\vec{X_2},\dots,\vec{X_d}\})
\end{align}
where $\vec{X_1},\vec{X_2},\dots,\vec{X_d}$ were the columns of $\vec{X}$

The dimension of the column space is atmost d.

Let $\vec{A}$ is of the order $m \times n$. If the dimension of the row space of $\vec{A}$ is r(Row Rank),then $\vec{A}=\vec{X}\vec{Y}$ for some $m \times r$ matrix $\vec{X}$ and $r \times m$ matrix $\vec{Y}$

Let $\vec{Y_1},\vec{Y_2},\dots,\vec{Y_r}$ be a basis for the row space of $\vec{A}$ and let $\vec{A_1},\vec{A_2},\dots,\vec{A_m}$ be the rows of $\vec{A}$.Then,
\begin{align}
    \vec{A_1}&=x_{11}\vec{Y_1}+x_{12}\vec{Y_2}+\dots+x_{1r}\vec{Y_r}\notag \\
\vec{A_2}&=x_{21}\vec{Y_1}+x_{22}\vec{Y_2}+\dots+x_{2r}\vec{Y_r}\notag \\
&\vdots \notag\\
\vec{A_m}&=x_{m1}\vec{Y_1}+x_{m2}\vec{Y_2}+\dots+x_{mr}\vec{Y_r}\notag 
\end{align}
Thus $\vec{A}=\vec{X}\vec{Y}$ where $\vec{Y}$ is the matrix with rows $\vec{Y_1},\dots,\vec{Y_r}$ and $\vec{X}$ is the matrix of coefficients $\vec{X}(i,j)=x_{ij}$.
\begin{align}
Column Rank \leq Row Rank \label{eqcr}
\end{align}
Let $\vec{X_1},\vec{X_2},\dots,\vec{X_r}$ be a basis for the column space of $\vec{A}$ and let $\vec{A_1},\vec{A_2},\dots,\vec{A_n}$ be the columns of $\vec{A}$.Then,
\begin{align}
    \vec{A_1}&=\vec{X_1}y_{11}+\vec{X_2}y_{21}+\dots+\vec{X_r}y_{r1}\notag \\
\vec{A_2}&=\vec{X_1}y_{12}+\vec{X_2}y_{22}+\dots+\vec{X_r}y_{r2}\notag \\
&\vdots \notag\\
\vec{A_n}&=\vec{X_1}y_{1n}+\vec{X_2}y_{2n}+\dots+\vec{X_r}y_{rn}\notag 
\end{align}
Thus $\vec{A}=\vec{X}\vec{Y}$ where $\vec{X}$ is the matrix with columns $\vec{X_1},\dots,\vec{X_r}$ and $\vec{Y}$ is the matrix of coefficients $\vec{Y}(i,j)=y_{ij}$.
\begin{align}
Row Rank \leq Column Rank \label{eqcr2}
\end{align}
From equations \eqref{eqcr} and \eqref{eqcr2},we get , 
\begin{align}
    Row Rank = Column Rank \label{rank}
\end{align}
From the rank definition and the above equations \eqref{rr},\eqref{cr} and \eqref{rank},
\begin{align}
    Rank(\vec{A})\leq min(m,n)\label{main}
\end{align}
\subsection{Properties}
For a matrix $\vec{A}$ of order $m\times n$,

(a) If m is less than n, then the rank of the matrix will be atmost m.

(b) If m is greater than n, then the rank of the matrix will be atmost n.

\section{Proof}
Given , 
Matrix $\vec{A}$ is of order $m \times n$ and 
Matrix $\vec{B}$ is of order $n \times m$
\begin{align}
n<m 
\end{align}
From equation \eqref{main},since given $n<m$,
\begin{align}
    Rank(\vec{A})\leq n\\
    Rank(\vec{B})\leq n
\end{align}
$\vec{A}^T$ will be of order $n \times m$

From equation \eqref{main},
\begin{align}
    Rank(\vec{A}^T) &\leq min(n,m) \\
    \implies Rank(\vec{A}^T) &= Rank(\vec{A})\label{eq}
\end{align}
The maximum possible rank of $\vec{A}$ and $\vec{B}$ is given by 
\begin{align}
    Rank(\vec{A})&=n\label{a}\\
    Rank(\vec{B})&=n\label{2}
\end{align}
$\vec{A}\vec{B}$ will be of order $m \times m$

Consider a vector $\vec{v}$,
\begin{align}
\vec{v} &\in col(\mathbf(\vec{A}\vec{B}))\label{11}\\
\vec{v}&=(\vec{A}\vec{B})\vec{x}\\
\vec{v}&=\vec{A}(\vec{B}\vec{x})\\
\vec{v} &\in col(\mathbf(\vec{A}))\label{12}
\end{align}
From equations \eqref{11} and \eqref{12},we could say that for every vector in the column space of $\vec{A}\vec{B}$ the same vector will be there in column space of $\vec{A}$ aswell.
\begin{align}
    Rank(\vec{A}\vec{B})&\leq Rank(\vec{A})\label{f1}
\end{align}
From equation \eqref{eq},
\begin{align}
 Rank(\vec{A}\vec{B})&=Rank((\vec{A}\vec{B})^T)\\
\implies Rank((\vec{A}\vec{B})^T)&=Rank(\vec{B}^T\vec{A}^T)
\end{align}
From equation \eqref{f1},
\begin{align}
    \implies Rank(\vec{B}^T\vec{A}^T)&\leq Rank(\vec{B}^T)\\
\implies Rank(\vec{B}^T\vec{A}^T)&\leq Rank(\vec{B})\\
\implies Rank(\vec{A}\vec{B})&\leq Rank(\vec{B})\label{f2}
\end{align}
From the equations \eqref{f1} and \eqref{f2},
\begin{align}
    Rank(\vec{A}\vec{B})&\leq min(Rank(\vec{A}),Rank(\vec{B}))\\
    Rank(\vec{A}\vec{B})&\leq n
\end{align}
The maximum possible rank of $\vec{A}\vec{B}$ of order $m\times m$ is given by 
\begin{align}
    Rank(\vec{A}\vec{B})= n < m\label{final}
\end{align}
From \eqref{final} we could say that $\vec{A}\vec{B}$ does not have a full rank and if the matrix does not have a full rank then it is not invertible.Hence, $\vec{A}\vec{B}$ is not invertible.
\begin{center}
    Hence Proved
\end{center}
\end{document}

